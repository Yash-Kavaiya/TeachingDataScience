%%%%%%%%%%%%%%%%%%%%%%%%%%%%%%%%%%%%%%%%%%%%%%%%%%%%%%%%%%%%%%%%%%%%%%%%%%%%%%%%%%
\begin{frame}[fragile]\frametitle{}
\begin{center}
{\Large Implementations}
\end{center}
\end{frame}

% %%%%%%%%%%%%%%%%%%%%%%%%%%%%%%%%%%%%%%%%%%%%%%%%%%%%%%%%%%%
% \begin{frame}[fragile]\frametitle{Demo}

% https://colab.research.google.com/drive/
% 12cdBWMpOfCxpiAS1zSqZRY66o84qMiTo?usp=sharing 


% \begin{center}
% \includegraphics[width=0.4\linewidth,keepaspectratio]{llm28}

% {\tiny (Ref: LlamaIndex: A Central Interface between LLM’s + your external data)}
% \end{center}
% \end{frame}

% %%%%%%%%%%%%%%%%%%%%%%%%%%%%%%%%%%%%%%%%%%%%%%%%%%%%%%%%%%%
\begin{frame}[fragile]\frametitle{LlamaIndex Chatbot: Overview}
    \begin{itemize}
        \item LlamaIndex connects LLMs to custom data sources
        \item Main steps: Import data, build index, query index
        \item Optional: Chunk documents, build sub-indices
        \item GPT model (e.g., text-davinci-003) handles responses
    \end{itemize}
\end{frame}

%%%%%%%%%%%%%%%%%%%%%%%%%%%%%%%%%%%%%%%%%%%%%%%%%%%%%%%%%%%
\begin{frame}[fragile]\frametitle{Install Required Libraries}
    \begin{itemize}
        \item Install libraries via pip:
        \item Set OpenAI API key:
    \end{itemize}
\begin{lstlisting}[language=Python]
pip install llama-index
pip install openai
import os
os.environ['OPENAI_API_KEY'] = 'API_KEY'
\end{lstlisting}
\end{frame}

%%%%%%%%%%%%%%%%%%%%%%%%%%%%%%%%%%%%%%%%%%%%%%%%%%%%%%%%%%%
\begin{frame}[fragile]\frametitle{Ingesting Local Text Data}

    \begin{itemize}
		\item Load .txt data stored in local folder
        \item Use built-in loaders for Wikipedia and YouTube
    \end{itemize}
	
\begin{lstlisting}[language=Python]
from llama_index import SimpleDirectoryReader
from llama_index import download_loader

documents_txt = SimpleDirectoryReader('data').load_data()

WikipediaReader = download_loader("WikipediaReader")
documents = WikipediaReader().load_data(pages=['Strawberry'])


YoutubeTranscriptReader = download_loader("YoutubeTranscriptReader")
documents_youtube = YoutubeTranscriptReader().load_data(
    ytlinks=['https://www.youtube.com/watch?v=EYXQmbZNhy8'])
\end{lstlisting}
\end{frame}

%%%%%%%%%%%%%%%%%%%%%%%%%%%%%%%%%%%%%%%%%%%%%%%%%%%%%%%%%%%
\begin{frame}[fragile]\frametitle{Constructing the Indices}
 Convert documents into GPT-based vector indexes
\begin{lstlisting}[language=Python]
from llama_index import GPTSimpleVectorIndex

index_txt = GPTSimpleVectorIndex.from_documents(documents_txt)
index_wiki = GPTSimpleVectorIndex.from_documents(documents)
index_youtube = GPTSimpleVectorIndex.from_documents(documents_youtube)
\end{lstlisting}
\end{frame}

%%%%%%%%%%%%%%%%%%%%%%%%%%%%%%%%%%%%%%%%%%%%%%%%%%%%%%%%%%%
\begin{frame}[fragile]\frametitle{Querying the Indices}
Simple querying using index.query()

\begin{lstlisting}[language=Python]
index_txt.query("Which fruit is the best?").response
index_wiki.query("Which countries produce strawberries?").response
index_youtube.query("how should I measure the flour?").response
\end{lstlisting}
\end{frame}

%%%%%%%%%%%%%%%%%%%%%%%%%%%%%%%%%%%%%%%%%%%%%%%%%%%%%%%%%%%
\begin{frame}[fragile]\frametitle{Saving Index Files}

Save indices to disk for use in a web app

\begin{lstlisting}[language=Python]
index_wiki.save_to_disk('index_wikepedia.json')
index_youtube.save_to_disk('index_video.json')
index_txt.save_to_disk('index_txt.json')
\end{lstlisting}
\end{frame}

%%%%%%%%%%%%%%%%%%%%%%%%%%%%%%%%%%%%%%%%%%%%%%%%%%%%%%%%%%%
\begin{frame}[fragile]\frametitle{Streamlit App: Main Structure}
    \begin{itemize}
        \item Build chatbot interface with Streamlit
        \item Load saved indices and select by user input
    \end{itemize}
\begin{lstlisting}[language=Python]
import streamlit as st
from llama_index import GPTSimpleVectorIndex
import os

@st.cache_resource
def load_indexes():
    index_document = GPTSimpleVectorIndex.load_from_disk('index_txt.json')
    index_video = GPTSimpleVectorIndex.load_from_disk('index_video.json')
    index_wikepedia = GPTSimpleVectorIndex.load_from_disk('index_wikepedia.json')
    return index_document, index_video, index_wikepedia
\end{lstlisting}
\end{frame}

%%%%%%%%%%%%%%%%%%%%%%%%%%%%%%%%%%%%%%%%%%%%%%%%%%%%%%%%%%%
\begin{frame}[fragile]\frametitle{Streamlit App: Interaction}
\begin{lstlisting}[language=Python]
def main():
    os.environ['OPENAI_API_KEY'] = 'API_KEY'
    index_document, index_video, index_wikepedia = load_indexes()

    st.header('Custom-Made Chatbots')
    data = st.selectbox('Data', (
        '.txt file (My favorite fruits)', 
        'Youtube Video (Vanilla Cake Recipe)', 
        'Wikipedia Article (Apple)'
    ))

    index = index_document if data.startswith('.txt') else (
        index_video if 'Youtube' in data else index_wikepedia)

    query = st.text_input('Enter Your Query')
    if st.button('Response'):
        st.write(index.query(query))
\end{lstlisting}
\end{frame}

%%%%%%%%%%%%%%%%%%%%%%%%%%%%%%%%%%%%%%%%%%%%%%%%%%%%%%%%%%%
\begin{frame}[fragile]\frametitle{Launching the Web App}

Run the chatbot app using Streamlit

\begin{lstlisting}[language=bash]
streamlit run app.py
\end{lstlisting}

\end{frame}





