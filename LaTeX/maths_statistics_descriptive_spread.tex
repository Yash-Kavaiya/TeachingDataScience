%%%%%%%%%%%%%%%%%%%%%%%%%%%%%%%%%%%%%%%%%%%%%%%%%%%%%%%%%%%%%%%%%%%%%%%%%%%%%%%%%%
\begin{frame}[fragile]\frametitle{}
\begin{center}
{\Large Measure of Spread}
\end{center}
\end{frame}

%%%%%%%%%%%%%%%%%%%%%%%%%%%%%%%%%%%%%%%%%%%%%%%%%%%%%%%%%%
\begin{frame}[fragile]\frametitle{Measure of Spread}	

In which example (below), the data is spread?

\begin{center}
\includegraphics[width=0.8\linewidth,keepaspectratio]{spread}
\end{center}

How do you quantify the spread?

\end{frame}


%%%%%%%%%%%%%%%%%%%%%%%%%%%%%%%%%%%%%%%%%%%%%%%%%%%%%%%%%%
\begin{frame}[fragile]\frametitle{Measure of Spread}	

\begin{itemize}
\item Range: The largest value minus the smallest value. Suffers from Outliers.
\item Semi-Interquartile range: One half of the difference between the 75th percentile and the 25th percentile. Not affected by Outliers.
\item Standard Deviation:	The square root of the average of the squared deviations from the mean
\end{itemize}

%\begin{center}
%\includegraphics[width=0.8\linewidth,keepaspectratio]{mode}
%\end{center}

\end{frame}



%%%%%%%%%%%%%%%%%%%%%%%%%%%%%%%%%%%%%%%%%%%%%%%%%%%%%%%%%%
\begin{frame}[fragile]\frametitle{Range}	
\begin{itemize}
\item Variation between the smallest and the largest values
\item Can be misleading if most values are concentrated, but a few values are extreme
\end{itemize}
$range(x) = max(x) - min(x)$

\end{frame}

%%%%%%%%%%%%%%%%%%%%%%%%%%%%%%%%%%%%%%%%%%%%%%%%%%%%%%%%%%
\begin{frame}[fragile]\frametitle{Range}	
\begin{center}
\includegraphics[width=0.6\linewidth,keepaspectratio]{da9}
\end{center}
Range = 11 -1 = 10

\begin{itemize}
\item A `quick and easy' indication of variability
\item No indication of dispersion within
\item Unstable, as depends ONLY on Outliers/Extremes
\end{itemize}
\code{Calculate and verify the answer}
\end{frame}

%%%%%%%%%%%%%%%%%%%%%%%%%%%%%%%%%%%%%%%%%%%%%%%%%%%%%%%%%%%%%%%%%%%%%%%%
\begin{frame}[fragile]\frametitle{Range}
Implement my\_range. It cannot be called as ``range'' is already there in Python, so a different name
\begin{lstlisting}
def my_range(datalist):
	:
	return min, max, diff

lst = [9,3,7,2,7,10,23,44,12,42,19,11,22,5,3,4,3,21,3]
min,max,diff = my_range(lst)
print("Range: Min {}, Max {}, Diff {}".format(min,max,diff))
\end{lstlisting}
\end{frame}

%%%%%%%%%%%%%%%%%%%%%%%%%%%%%%%%%%%%%%%%%%%%%%%%%%%%%%%%%%
\begin{frame}[fragile]\frametitle{Range}
\begin{lstlisting}
def my_range(abclist):
    smallest = abclist[0]
    largest = abclist[0]
    range_of_values = 0
    for item in abclist[1:]:
        if item < smallest:
            smallest = item
        elif item > largest:
            largest = item
    range_of_values = largest - smallest
    return smallest, largest, range_of_values

Range: Min 2, Max 44, Diff 42
\end{lstlisting}
\end{frame}

%%%%%%%%%%%%%%%%%%%%%%%%%%%%%%%%%%%%%%%%%%%%%%%%%%%%%%%%%%
\begin{frame}[fragile]\frametitle{Range}
min max functions are available on list
\begin{lstlisting}
def my_range2(x):
	return max(x) - min(x)

diff = my_range2(lst)
print("Range: {}".format(diff))	
Range: 42
\end{lstlisting}

\end{frame}



% %%%%%%%%%%%%%%%%%%%%%%%%%%%%%%%%%%%%%%%%%%%%%%%%%%%%%%%%%%
% \begin{frame}[fragile]\frametitle{Range}	
% % \begin{lstlisting}
% % def range_min_max(abclist):
    % % smallest = abclist[0]
    % % largest = abclist[0]
    % % range_of_values = 0
    % % for item in abclist[1:]:
        % % if item < smallest:
            % % smallest = item
        % % elif item > largest:
            % % largest = item
    % % range_of_values = largest - smallest
    % % return smallest, largest, range_of_values
% % \end{lstlisting}

% %\code{Result? 58.5,51.5,49}
% \end{frame}






%%%%%%%%%%%%%%%%%%%%%%%%%%%%%%%%%%%%%%%%%%%%%%%%%%%%%%%%%%
\begin{frame}[fragile]\frametitle{Percentiles}	
\begin{itemize}
\item For ordered data, percentile is useful.
\item Given an ordinal or continuous attribute x and a number p between 0 and 100, the pth percentile $x_p$ is a value of x such that p\% of the observed values are less than $x_p$.
\item Example: the 75th percentile is the value such that 75\% of all values are less than it.
\end{itemize}

\end{frame}


%%%%%%%%%%%%%%%%%%%%%%%%%%%%%%%%%%%%%%%%%%%%%%%%%%%%%%%%%%
\begin{frame}[fragile]\frametitle{Quantile}	
Quantile can be of any number between 0 to 1. Quartiles are about quarters so they are quantiles of 0.25, 0.5,0.75

Quantiles are cut points in set of data. They can represent the bottom
ten percent of the data or the top 75\% or any \% from 0 to 100.
% \begin{lstlisting}
% def quantile(datalist,num):
	% index = int(num * len(datalist)) # slicing parameter
	% if num > .5:
		% return sorted(datalist)[index:]
	% else:
		% return sorted(datalist)[:index]
% \end{lstlisting}
% quantile(lst,0.10)
% quantile(lst,0.25)
% quantile(lst,0.75)
% quantile(lst,0.90)
% \code{Result?[2],[2,3,3,3],[21,22,23,42,44],[42,44] }
\end{frame}


%%%%%%%%%%%%%%%%%%%%%%%%%%%%%%%%%%%%%%%%%%%%%%%%%%%%%%%%%%%%%%%%%%%%%%%%
\begin{frame}[fragile]\frametitle{Quantile}
Implement quantile
\begin{lstlisting}
def quantile(datalist):
	:
	return q

def interquartile_range(x):
	:
	return iqr
	
lst = [9,3,7,2,7,10,23,44,12,42,19,11,22,5,3,4,3,21,3]
result1 = quantile(lst,0.10)
result2 = quantile(lst,0.25)
result3 = quantile(lst,0.75)
result4 = quantile(lst,0.90)
result5 = interquartile_range(lst)
print("Q10 {}, Q25 {}, Q50 {} Q90 {} IQR {}".format(result1,result2,result3,result4,result5))
\end{lstlisting}
\end{frame}


%%%%%%%%%%%%%%%%%%%%%%%%%%%%%%%%%%%%%%%%%%%%%%%%%%%%%%%%%%
\begin{frame}[fragile]\frametitle{Quantile}
\begin{lstlisting}
def quantile(datalist,num):
    index = int(num * len(datalist)) # slicing parameter
    return sorted(datalist)[index]
    # For values :
    # if num > .5:
    #     return sorted(datalist)[index:]
    # else:
    #     return sorted(datalist)[:index]

def interquartile_range(x):
    return	quantile(x, 0.75) - quantile(x, 0.25)

# Q10 [2], Q25 [2, 3, 3, 3], Q50 [21, 22, 23, 42, 44] Q90 [42, 44]
Q10 3, Q25 3, Q50 21 Q90 42 IQR 18
\end{lstlisting}
\end{frame}

%%%%%%%%%%%%%%%%%%%%%%%%%%%%%%%%%%%%%%%%%%%%%%%%%%%%%%%%%%
\begin{frame}[fragile]\frametitle{Semi-Interquartile Range}	
\begin{itemize}
\item Quartiles are Quantiles at 25\% and 75\%.
\item Inter Quartile Range (IQR) is between 25\% and 75\%.
\item More resistant to extreme values than the range
\item Does not utilize all the values in the data or set for its computation
\item If small, the values are concentrated near the median
\end{itemize}
\end{frame}

%%%%%%%%%%%%%%%%%%%%%%%%%%%%%%%%%%%%%%%%%%%%%%%%%%%%%%%%%%
\begin{frame}[fragile]\frametitle{Semi-Interquartile Range}	
\begin{itemize}
\item 75th percentile: the value in the date set which is exceeded by 75\% of the total number of items in the set
\item 25 x (0.75) = 18.75
\item 18.75 : rank of the 75th percentile
\item 18th  and 19th items, both  8
\item 75th percentile = 8
\end{itemize}
\begin{center}
\includegraphics[width=0.4\linewidth,keepaspectratio]{da13}
\end{center}
\end{frame}


%%%%%%%%%%%%%%%%%%%%%%%%%%%%%%%%%%%%%%%%%%%%%%%%%%%%%%%%%%
\begin{frame}[fragile]\frametitle{Semi-Interquartile Range}	
\begin{itemize}
\item 25th percentile: the value in the date set which is exceeded by 25\% of the total number of items in the set
\item 25 x (0.25) = 6.25
\item 6.25 : rank of the 25th percentile
\item 6th item = 3 and 7th item = 4
\item 25th percentile = 3 + (0.25)(4-3)
\item 25th percentile = 3.25
\end{itemize}
\begin{center}
\includegraphics[width=0.4\linewidth,keepaspectratio]{da14}
\end{center}
\code{Calculate and verify the answer}
\end{frame}


%%%%%%%%%%%%%%%%%%%%%%%%%%%%%%%%%%%%%%%%%%%%%%%%%%%%%%%%%%
\begin{frame}[fragile]\frametitle{Semi-Interquartile Range}	
\begin{itemize}
\item 75th percentile = 8
\item 25th percentile = 3.25
\item SIQR =1/2 (8 - 3.25)
\item SIQR = 2.375
\item Semi-interquartile range = 2.375
\end{itemize}

\begin{center}
\includegraphics[width=0.8\linewidth,keepaspectratio]{da10}
\end{center}
%\code{Calculate and verify the answer}
\end{frame}



%%%%%%%%%%%%%%%%%%%%%%%%%%%%%%%%%%%%%%%%%%%%%%%%%%%%%%%%%%
\begin{frame}[fragile]\frametitle{Standard Deviation}	
\begin{itemize}
\item How far each value if from the mean
\item Uses all the values in the data for its computation
\item If small, the values are concentrated near the mean.
\item If LARGE, the values are scattered widely about the mean
\item z score: how many std deviations from the mean.
\end{itemize}
\begin{center}
\includegraphics[width=\linewidth,keepaspectratio]{da16}
\end{center}
\end{frame}


%%%%%%%%%%%%%%%%%%%%%%%%%%%%%%%%%%%%%%%%%%%%%%%%%%%%%%%%%%%%%%%%%%%%%%%%
\begin{frame}[fragile]\frametitle{Variance, Standard Deviation}
Implement variance and standard deviation.
\begin{center}
\includegraphics[width=0.5\linewidth,keepaspectratio]{da16}
\end{center}

\begin{lstlisting}
def variance(datalist):
	:
	return v

def std_dev(datalist):
	:
	return s

lst = [9,3,7,2,7,10,23,44,12,42,19,11,22,5,3,4,3,21,3]
result1 = variance(lst)
result2 = std_dev(lst)
print("Variance {}, Std Dev {}".format(result1,result2))
\end{lstlisting}
\end{frame}


% %%%%%%%%%%%%%%%%%%%%%%%%%%%%%%%%%%%%%%%%%%%%%%%%%%%%%%%%%%
% \begin{frame}[fragile]\frametitle{Calculate Deviation}	
% \begin{lstlisting}
% def mean(datalist):
    % total = 0
    % mean = 0
    % for item in datalist:
        % total += item
    % mean = total / float(len(datalist))
    % return mean
 
% def avg_dev(thislist):
    % average = mean(thislist)
    % sum_of_dev = 0
    % avg_dev = 0
    % for item in thislist:
        % sum_of_dev += abs((average - item))
    % avg_dev = sum_of_dev / len(thislist)
    % return avg_dev
% \end{lstlisting}
% \end{frame}


%%%%%%%%%%%%%%%%%%%%%%%%%%%%%%%%%%%%%%%%%%%%%%%%%%%%%%%%%%
\begin{frame}[fragile]\frametitle{Standard Deviation}	
To reparametrize Covariance value between -1 and 1, need to divide by std devs. Empirical Rule for symmetric bell-shaped distributions
\begin{itemize}
\item About 68\% of the values will lie within 1 standard deviation of the mean
\item About 95\% of the values will lie Within 2 standard deviation 
\item About 99.7\% of the values will lie within 3 standard deviation of the mean
\end{itemize}
$variance(x) = s_x^2 = 1/(n-1)\sum (x_i - \bar{x})^2$
$sd(x) = s_x = \sqrt{1/(n-1)\sum (x_i - \bar{x})^2}$
\end{frame}

% %%%%%%%%%%%%%%%%%%%%%%%%%%%%%%%%%%%%%%%%%%%%%%%%%%%%%%%%%%
% \begin{frame}[fragile]\frametitle{Calculate Deviation}	
% \begin{lstlisting}
% def variance(thatlist):
    % average = mean(thatlist)
    % sum_of_sqrt_dev = 0
    % variance = 0
    % for item in thatlist:
        % sum_of_sqrt_dev += (average - item) ** 2
    % variance = sum_of_sqrt_dev / len(thatlist)
    % return variance
 
% def std_dev(anotherlist):
    % std_dev = variance(anotherlist) ** 0.5
    % return std_dev
% \end{lstlisting}
% \end{frame}


%%%%%%%%%%%%%%%%%%%%%%%%%%%%%%%%%%%%%%%%%%%%%%%%%%%%%%%%%%
\begin{frame}[fragile]\frametitle{Variance, Standard Deviation}
\begin{lstlisting}
def de_mean(x):
	"""translate x by subtracting its mean"""
	x_bar = mean(x)
	return	[x_i - x_bar for x_i in x]

def sum_of_squares(diffs):
	sum_of_squares = 0
	for df in diffs:
          sum_of_squares += (df) ** 2
     return sum_of_squares

def variance(x):
	"""assumes x has at	least two elements"""
	n = len(x)
	deviations = de_mean(x)
	return	sum_of_squares(deviations) / (n - 1)
 
def std_dev(anotherlist):
	std_dev = variance(anotherlist) ** 0.5
	return std_dev

Variance 158.36257309941527, Std Dev 12.584219208970229
\end{lstlisting}
\end{frame}


%%%%%%%%%%%%%%%%%%%%%%%%%%%%%%%%%%%%%%%%%%%%%%%%%%%%%%%%%%%
%\begin{frame}[fragile]\frametitle{Standard Deviation}	
%Mean = 6.12
%Standard Deviation = 2.934
%\begin{center}
%\includegraphics[width=0.6\linewidth,keepaspectratio]{da11}
%\end{center}
%\code{Calculate and verify the answer}
%\end{frame}

%%%%%%%%%%%%%%%%%%%%%%%%%%%%%%%%%%%%%%%%%%%%%%%%%%%%%%%%%%
\begin{frame}[fragile]\frametitle{Standard Deviation}	
Standard Deviation = 2.934
\begin{center}
\includegraphics[width=\linewidth,keepaspectratio]{da12}
\end{center}
\end{frame}

%%%%%%%%%%%%%%%%%%%%%%%%%%%%%%%%%%%%%%%%%%%%%%%%%%%%%%%%%%%%%%%%%%%%%%%%%%%%%%%%%%
\begin{frame}[fragile]\frametitle{}
\begin{center}
{\Large Descriptive Statistics Exercise}
\end{center}
\end{frame}

%%%%%%%%%%%%%%%%%%%%%%%%%%%%%%%%%%%%%%%%%%%%%%%%%%%%%%%%%%
\begin{frame}[fragile]\frametitle{Exercise}	

\code{Result?(45,82,37),11.25,161.45,12.7062 }


\begin{lstlisting}
crater_diameter = [46, 51, 49, 82, 74, 63, 49, 70, 48, 47, 79, 48, 52, 55, 49, 51, 58, 82, 72, 45]
 
print range_min_max(crater_diameter)
print avg_dev(crater_diameter)
print variance(crater_diameter)
print std_dev(crater_diameter)
\end{lstlisting}
\end{frame}


%%%%%%%%%%%%%%%%%%%%%%%%%%%%%%%%%%%%%%%%%%%%%%%%%%%%%%%%%%
\begin{frame}[fragile]\frametitle{Exercise}	
Find the mean, median, range and standard deviation for the following set of data:

\lstinline|2.8, 8.7, 0.7, 4.9, 3.4, 2.1 & 4.0|
\end{frame}

%%%%%%%%%%%%%%%%%%%%%%%%%%%%%%%%%%%%%%%%%%%%%%%%%%%%%%%%%%
\begin{frame}[fragile]\frametitle{Exercise}	
Find the mean, median, range and standard deviation for the following set of data:

\begin{center}
\includegraphics[width=0.6\linewidth,keepaspectratio]{da17}
\end{center}
\end{frame}

%
%%%%%%%%%%%%%%%%%%%%%%%%%%%%%%%%%%%%%%%%%%%%%%%%%%%%%%%%%%%%%%%%%%%%%%%%
\begin{frame}[fragile]\frametitle{Difference between Standard Deviation and Standard Error}
\begin{columns}
    \begin{column}[T]{0.6\linewidth}
	\begin{itemize}
	\item For a set of normally distributed observations you have mean and standard deviation.
	\item If you do this for different samples, you get their own respective means and standard deviations.
	\end{itemize}

    \end{column}
    \begin{column}[T]{0.4\linewidth}
      \begin{center}
      \includegraphics[width=\linewidth,keepaspectratio]{statq28}
	  
	  \includegraphics[width=\linewidth,keepaspectratio]{statq29}
	   
	  	\end{center}
    \end{column}

  \end{columns}
  
\tiny{(Ref: StatQuest: Difference between Standard Deviation and Standard Error - Josh Starmer )}
\end{frame}

%%%%%%%%%%%%%%%%%%%%%%%%%%%%%%%%%%%%%%%%%%%%%%%%%%%%%%%%%%%%%%%%%%%%%%%%
\begin{frame}[fragile]\frametitle{Difference between Standard Deviation and Standard Error}

	\begin{itemize}
	\item Plotting those sample means, and sample standard deviations, can form another (meta?) distribution
	\item Standard deviation of this meta distribution is called Standard Error
	\end{itemize}

      \begin{center}
      \includegraphics[width=0.8\linewidth,keepaspectratio]{statq30}
	\end{center}

  
\tiny{(Ref: StatQuest: Difference between Standard Deviation and Standard Error - Josh Starmer )}
\end{frame}
