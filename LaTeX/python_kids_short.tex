%%%%%%%%%%%%%%%%%%%%%%%%%%%%%%%%%%%%%%%%%%%%%%%%%%%%%%%%%%%%%%%%%%%%%%%%%%%%%%%%%%
\begin{frame}[fragile]\frametitle{}
\begin{center}
{\Large Kid's Play}
\end{center}
\end{frame}



%%%%%%%%%%%%%%%%%%%%%%%%%%%%%%%%%%%%%%%%%%%%%%%%%%%%%%%%%%%%%%%%%%%%%%%%%%%%%%%%%%%
\begin{frame}[fragile]\frametitle{Finding Phone Number}
\begin{itemize}
\item Now a days, many email/chat programs find and highlight phone numbers in a message, so that you can call the number directly by clicking it.
\item Imagine you need to write that program: Finding phone numbers in a message.
\item How would you go about it?
\item Psuedo code?
\end{itemize}
\end{frame}

%%%%%%%%%%%%%%%%%%%%%%%%%%%%%%%%%%%%%%%%%%%%%%%%%%%%%%%%%%%%%%%%%%%%%%%%%%%%%%%%%%%
\begin{frame}[fragile]\frametitle{Finding Phone Number}
%\adjustbox{valign=t}{
%\begin{minipage}{0.45\linewidth}
\begin{itemize}
\item Split the message with ` ` (space) to get words
\item for each word, check if it is a phone number of format, xxx-xxx-xxxx (US style, for now)
%\item One possible implementation of $isPhoneNumber()$ is shown on the right	
\end{itemize}
%Want something better?
%\end{minipage}
%}
%\hfill
%\adjustbox{valign=t}{
%\begin{minipage}{0.45\linewidth}
\begin{lstlisting}
def isPhoneNumber(text):
	if len(text) != 12:
           return False
	for i in range(0, 3):
		if not text[i].isdecimal():
			return False
	if text[3] != '-':
		return False
	for i in range(4, 7):
		if not text[i].isdecimal():
               	return False
	if text[7] != '-':
		return False
	for i in range(8, 12):
		if not text[i].isdecimal():
			return False
	return True
\end{lstlisting}
%\end{minipage}
%}
\end{frame}

%%%%%%%%%%%%%%%%%%%%%%%%%%%%%%%%%%%%%%%%%%%%%%%%%%%%%%%%%%%%%%%%%%%%%%%%%%%%%%%%%%%
\begin{frame}[fragile]\frametitle{Regular Expressions}
\begin{itemize}
\item Regular expressions (regex), are descriptions for a pattern.
\item \lstinline{\d} stands for any digit from 0 to 9.
\item \lstinline{\d\d\d-\d\d\d-\d\d\d\d} represents the phone number.
\item Same, but slightly shorter regex is \lstinline|\d{3}-\d{3}-\d{4}|
\end{itemize}

Prints ``Phone number found: 415-555-4242'' \\
Becomes as easy as Kid's play?

\begin{lstlisting}
import re

message = 'Call me at 415-555-1011 tomorrow.'
phoneNumRegex = re.compile(r'\d\d\d-\d\d\d-\d\d\d\d')
mo = phoneNumRegex.search(message)
print('Phone number found: ' + mo.group())
\end{lstlisting}

\end{frame}

%%%%%%%%%%%%%%%%%%%%%%%%%%%%%%%%%%%%%%%%%%%%%%%%%%%%%%%%%%%%%%%%%%%%%%%%%%%%%%%%%%%
\begin{frame}[fragile]\frametitle{Google Search}
My Steps (most likely your's too):
\begin{itemize}
\item Type search term in the box of $google.com$.
\item Middle-click top few search result links to open in new tabs.
\item Doing this one by one is tedious.
\item Any way to automate this?
\end{itemize}
\end{frame}


%%%%%%%%%%%%%%%%%%%%%%%%%%%%%%%%%%%%%%%%%%%%%%%%%%%%%%%%%%%%%%%%%%%%%%%%%%%%%%%%%%%
\begin{frame}[fragile]\frametitle{Using Python}
\begin{itemize}
\item Gets search keywords from the command line arguments.
\item Retrieves the search results page.
\item Opens a browser tab for each result.
\end{itemize}

Opens result links in new browser tabs. \\
Becomes as easy as Kid's play?

\begin{lstlisting}
import requests, sys, webbrowser, bs4

res = requests.get('http://google.com/search?q=' + \
					 ' '.join(sys.argv[1:]))
soup = bs4.BeautifulSoup(res.text)
linkElems = soup.select('.r a')
numOpen = min(5, len(linkElems))
for i in range(numOpen):
    webbrowser.open('http://google.com' +\
    		 linkElems[i].get('href'))
\end{lstlisting}

\end{frame}

%%%%%%%%%%%%%%%%%%%%%%%%%%%%%%%%%%%%%%%%%%%%%%%%%%%%%%%%%%%%%%%%%%%%%%%%%%%%%%%%%%%
\begin{frame}[fragile]\frametitle{Fetching Weather info}
\begin{itemize}
\item Reads the requested locations from the command line.
\item Downloads JSON weather data from OpenWeatherMap.org.
\item Prints the weather for today and the next two days.
\end{itemize}

Becomes as easy as Kid's play?


\begin{lstlisting}
import json, requests, sys

location = ' '.join(sys.argv[1:])
url =`http://api.openweathermap.org/data/2.5/forecast/daily?q={}&cnt=3'.format(location)
response = requests.get(url)

weatherData = json.loads(response.text)
w = weatherData['list']

print(w[0]['weather'][0]['main'], '-', w[0]['weather'][0]['description'])
print(w[1]['weather'][0]['main'], '-', w[1]['weather'][0]['description'])
print(w[2]['weather'][0]['main'], '-', w[2]['weather'][0]['description'])
\end{lstlisting}
\end{frame}
