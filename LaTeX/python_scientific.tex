%%%%%%%%%%%%%%%%%%%%%%%%%%%%%%%%%%%%%%%%%%%%%%%%%%%%%%%%%%%%%%%%%%%%%%%%%%%%%%%%%%
\begin{frame}[fragile]\frametitle{}
\begin{center}
{\Large Packages: Scientific}
\end{center}
\end{frame}

%%%%%%%%%%%%%%%%%%%%%%%%%%%%%%%%%%%%%%%%%%%%%%%%%%%%%%%%%%%%%%%%%%%%%%%%%%%%%%%%%%%
\begin{frame}[fragile]\frametitle{NumPy: linear algebra package}

\begin{itemize}
\item     \href{http://www.numpy.org/}{NumPy} is a package for linear algebra
  and advanced mathematics in Python.

\item    
  It provides a \emph{fast} implementation of multidimensional
  numerical arrays (C/FORTRAN like), vectors, matrices, tensors and
  operations on them.

\item    
  \emph{Use it if:} you long for MATLAB core features.

  \item Reference: {\footnotesize\url{http://www.numpy.org/}}
  \item Examples: {\footnotesize\url{http://wiki.scipy.org/Numpy_Example_List}}
  \end{itemize}
\end{frame}

%%%%%%%%%%%%%%%%%%%%%%%%%%%%%%%%%%%%%%%%%%%%%%%%%%%%%%%%%%%%%%%%%%%%%%%%%%%%%%%%%%%
\begin{frame}[fragile]\frametitle{NumPy vs. Matlab vs. Bytecode}

    Comparison of $\nabla^2 u = 0$ solvers ($500\times500$, 100 iterations):
    
    \begin{columns}
        \column{0.4\linewidth}
        \begin{table}
            \begin{tabular}{|c|c|}
                \hline
                Platform & Time (s) \\
                \hline \hline
                Python & $\sim$1500.0 \\
                \hline
                NumPy & 29.3 \\
                \hline
                Matlab & $\sim$29.0 \\
                \hline
                Octave & $\sim$60.0 \\
                \hline
                Blitz (C++) & 9.5 \\
                \hline
                Fortran & 2.5 \\
                \hline
                C & 2.2 \\
                \hline
            \end{tabular}
        \end{table}
        \column{0.6\linewidth}
        \begin{center}
            \includegraphics[width=\linewidth]{numpy_perf1.png}
        \end{center}
    \end{columns}

%    (Probably using MKL)
%
%    {\tiny \href{http://www.scipy.org/PerformancePython}{http://www.scipy.org/PerformancePython}}\\
%    {\tiny \href{http://lbolla.info/blog/2007/04/11/numerical-computing-matlab-vs-pythonnumpyweave/}{http://lbolla.info/blog/2007/04/11/numerical-computing-matlab-vs-pythonnumpyweave/}}
\end{frame}

%%%%%%%%%%%%%%%%%%%%%%%%%%%%%%%%%%%%%%%%%%%%%%%%%%%%%%%%%%%%%%%%%%%%%%%%%%%%%%%%%%%
\begin{frame}[fragile]\frametitle{NumPy Array Generation}

    Convert a list to a NumPy array.     NumPy arrays have values and a data type (\textit{dtype}).

    \begin{lstlisting}
>>> import numpy as np
>>> list1 = [1, 2, 3, 4, 5]
>>> x = np.array(list1)
>>> x
array([1, 2, 3, 4, 5])
>>> x.dtype
dtype('int64')
    \end{lstlisting}
\end{frame}

%%%%%%%%%%%%%%%%%%%%%%%%%%%%%%%%%%%%%%%%%%%%%%%%%%%%%%%%%%%%%%%%%%%%%%%%%%%%%%%%%%%
\begin{frame}[fragile]
    \frametitle{NumPy Array Generation}

    \lstinline|arange()| is similar to \lstinline|range()| for lists:
    \begin{lstlisting}
>>> np.arange(5)
array([0, 1, 2, 3, 4])
>>> np.arange(5.)
array([0., 1., 2., 3., 4.])
    \end{lstlisting}
    
    \lstinline|zeros| for pre-allocation:
    \begin{lstlisting}
>>> np.zeros([2, 3])    # Input is a list
array([[ 0., 0., 0.],
       [ 0., 0., 0.]])
    \end{lstlisting}
\end{frame}

%%%%%%%%%%%%%%%%%%%%%%%%%%%%%%%%%%%%%%%%%%%%%%%%%%%%%%%%%%%%%%%%%%%%%%%%%%%%%%%%%%%
\begin{frame}[fragile]
    \frametitle{NumPy Grid Generation}
   
    Suppose you want to construct a numerical grid.\\
    e.g. $x = x_0 + j \Delta x$, \ $\Delta x = (x_1 - x_0) / N$
    \\~\\
    \lstinline|arange()| uses $\Delta x$ and excludes $x_1$:
    \begin{lstlisting}
>>> x = np.arange(3., 6., 0.5)
array([ 3. ,  3.5,  4. ,  4.5,  5. ,  5.5])
    \end{lstlisting}
    \lstinline|linspace()| uses $N+1$ and includes $x_1$:
    \begin{lstlisting}
>>> x = np.linspace(0., 1., 6)
array([ 0. ,  0.2,  0.4,  0.6,  0.8,  1. ])
    \end{lstlisting}
\end{frame}

%%%%%%%%%%%%%%%%%%%%%%%%%%%%%%%%%%%%%%%%%%%%%%%%%%%%%%%%%%%%%%%%%%%%%%%%%%%%%%%%%%%
\begin{frame}[fragile]
    \frametitle{NumPy Arithmetic}
    
    All NumPy operations are vectorized
	
    \begin{lstlisting}
>>> x = np.linspace(0., 4., 5)
>>> y = np.linspace(-2., -2., 5)
>>> 2*x
array([ 0.,  2.,  4.,  6.,  8.])
>>> x+y
array([-2.,  0.,  2.,  4.,  6.])

% Most mathematical functions are supported:

>>> np.exp(x)
>>> np.arctan(x)
>>> np.pi
    \end{lstlisting}
\end{frame}

%%%%%%%%%%%%%%%%%%%%%%%%%%%%%%%%%%%%%%%%%%%%%%%%%%%%%%%%%%%%%%%%%%%%%%%%%%%%%%%%%%%
\begin{frame}[fragile]
    \frametitle{Inefficient Arithmetic}

    Never do this:
    \begin{lstlisting}
>>> for i in range(10):
        z[i] = x[i] + y[i]
		
    % It will send 10 separate jobs to the C libraries.
    \\~\\
    % Always try to do vectorized calculations:
>>> z = x+y

	%     It only sends one job.
    \end{lstlisting}
\end{frame}

%%%%%%%%%%%%%%%%%%%%%%%%%%%%%%%%%%%%%%%%%%%%%%%%%%%%%%%%%%%%%%%%%%%%%%%%%%%%%%%%%%%
\begin{frame}[fragile]\frametitle{Arithmetic Quiz}

    Use \lstinline|np.mean()| to estimate the mean value of $f(x)$ over $[-1,1]$:

    \begin{columns}
        \column{0.5\linewidth}        
            \[ \frac{1}{2} \int_{-1}^{1} f(x) \; dx \]
        \column{0.5\linewidth}    
            \begin{itemize}
                \item $f(x) = \sin x$
                \item $f(x) = 1/(1 + x^2)$
                \item $f(x) = x^2 \exp(-x^2)$
            \end{itemize}
    \end{columns}
    \begin{lstlisting}
import numpy as np

N = 1001    # (N-1) intervals
x = np.linspace(-1., 1., N)

print(np.mean(np.sin(x)))
print(np.mean(1/(1 + x**2)))
print(np.mean(x**2 * np.exp(-x**2)))
    \end{lstlisting}
	
% \[PTO \ldots\]
\end{frame}

%%%%%%%%%%%%%%%%%%%%%%%%%%%%%%%%%%%%%%%%%%%%%%%%%%%%%%%%%%%%%%%%%%%%%%%%%%%%%%%%%%%
\begin{frame}[fragile]\frametitle{Arithmetic Quiz}

Note: the formulation stated earlier is an approximation to original approximation suggested in Trapezoidal rule

\begin{align*}
\int_{a}^{b} f(x)\, dx & \approx \frac{\Delta x}{2} \sum_{k=1}^{N} \left( f(x_{k-1}) + f(x_{k}) \right) \\
&= \frac{\Delta x}{2} ( f(x_0) + 2f(x_1) + 2f(x_2) + \dotsb + 2f(x_{N-1}) + f(x_N) )\\
&= \frac{\Delta x}{2} \left( f(x_0) + 2\sum_{k=1}^{N-1} f(x_k) + f(x_N) \right) 
\end{align*}

(Ref: https://en.wikipedia.org/wiki/Trapezoidal\_rule)
\end{frame}


%%%%%%%%%%%%%%%%%%%%%%%%%%%%%%%%%%%%%%%%%%%%%%%%%%%%%%%%%%%%%%%%%%%%%%%%%%%%%%%%%%%
\begin{frame}[fragile]\frametitle{Multidimensional Arrays}

    \begin{lstlisting}
%     NumPy supports multidimensional arrays:

>>> x = np.zeros((3,4))
>>> x
array([[ 0.,  0.,  0.,  0.],
       [ 0.,  0.,  0.,  0.],
       [ 0.,  0.,  0.,  0.]])
>>> x[:,0]
array([ 0.,  0.,  0.])
>>> x[0] # or x[0,:]
array([ 0.,  0.,  0., 0.])

%    NumPy arrays support comma-separated dimension indexing

    \end{lstlisting}

\end{frame}

%%%%%%%%%%%%%%%%%%%%%%%%%%%%%%%%%%%%%%%%%%%%%%%%%%%%%%%%%%%%%%%%%%%%%%%%%%%%%%%%%%%%
%\begin{frame}[fragile]\frametitle{Grid Arrays}
%
%    \lstinline|np.meshgrid()| lets you generate 2D grids from 1D axes:
%    \begin{lstlisting}
%>>> x_axis = np.linspace(-1., 1., 11)
%>>> y_axis = np.linspace(0., 1., 6)
%>>> x, y = np.meshgrid(x_axis, y_axis)
%    \end{lstlisting}
%    Inspect the shape of \lstinline|x| and \lstinline|y|. How are dimensions arranged?
%
%\end{frame}
%
%%%%%%%%%%%%%%%%%%%%%%%%%%%%%%%%%%%%%%%%%%%%%%%%%%%%%%%%%%%%%%%%%%%%%%%%%%%%%%%%%%%%
%\begin{frame}[fragile]\frametitle{Multidimensional Quiz}
%
%    Construct $f(x,y) = \sin x \cosh y$ on $(x,y) = [-1,1]\times[0,1]$.
%    \\~\\
%    How would you compute $df/dx$ and $df/dy$?
%%    \onslide<2->{
%%        \lstinputlisting{scripts/2d_grid.py}
%%    }
%    \begin{lstlisting}
%import numpy as np
%
%x_ax = np.linspace(-1., 1., 11)
%y_ax = np.linspace(0., 1., 6)
%x, y = np.meshgrid(x_ax, y_ax)
%
%f = np.sin(x) * np.cosh(y)
%print f   # grid is (y, x)!
%df_x = f[:, 1:] - f[:, :-1]
%df_y = f[1:, :] - f[:-1, :]
%    \end{lstlisting}
%\end{frame}

%%%%%%%%%%%%%%%%%%%%%%%%%%%%%%%%%%%%%%%%%%%%%%%%%%%%%%%%%%%%%%%%%%%%%%%%%%%%%%%%%%%
\begin{frame}[fragile]\frametitle{Reshaping}
    Two ways to reshape an array:
    \\~\\
    \lstinline|reshape()| outputs a new reshaped array:
    \begin{lstlisting}
>>> x = np.arange(12)
>>> x.reshape(3,4)
    \end{lstlisting}
%    
%    \lstinline|resize()| (or \lstinline|x.shape|) changes a shape:
%    \begin{lstlisting}
%>>> x.resize(6,2)
%>>> x
%>>> x.shape = 2,6
%>>> x
%    \end{lstlisting}
\end{frame}

%%%%%%%%%%%%%%%%%%%%%%%%%%%%%%%%%%%%%%%%%%%%%%%%%%%%%%%%%%%%%%%%%%%%%%%%%%%%%%%%%%%
\begin{frame}[fragile]\frametitle{Broadcasting}

    \begin{lstlisting}
%    Arithmetic usually requires arrays to be the same shape:
	
>>> x = np.arange(12).reshape(3,4)
>>> y = np.arange(12).reshape(4,3)
>>> x*y   # Does this work?

	%    But broadcasting will copy outer dimensions inward:

>>> x = np.ones(12).reshape(3,4)
>>> y = np.arange(4)
>>> x*y     # Outer dimension matches
%    As long as the last dimensions match, you can broadcast.

    \end{lstlisting}
\end{frame}

%%%%%%%%%%%%%%%%%%%%%%%%%%%%%%%%%%%%%%%%%%%%%%%%%%%%%%%%%%%%%%%%%%%%%%%%%%%%%%%%%%%
\begin{frame}[fragile]\frametitle{Extending Array Dimensions}

    What if you need to multiply along the first dimension?
    \begin{lstlisting}
>>> x = np.ones((3,4))
>>> y = np.arange(3)
>>> x * y                   # Won't work
>>> x * y[:, np.newaxis]    # Works!
    \end{lstlisting}
    \lstinline|np.newaxis| extends any missing dimension!
    \\~\\
    (Also see \lstinline|np.tile| and \lstinline|np.repeat|)
\end{frame}

%%%%%%%%%%%%%%%%%%%%%%%%%%%%%%%%%%%%%%%%%%%%%%%%%%%%%%%%%%%%%%%%%%%%%%%%%%%%%%%%%%%
\begin{frame}[fragile]\frametitle{Combining Arrays}

    Combine two arrays along some dimension:
    \begin{lstlisting}
>>> x = np.arange(5)
>>> y = np.arange(5)
>>> np.hstack((x,y))    # Stack on axis=0
>>> np.vstack((x,y))    # Stack on axis=1

# Use \lstinline|np.concatenate| for higher dimensions:


>>> x = np.ones((4,3,2))
>>> y = np.ones((4,3,1))
>>> np.concatenate((x,y),axis=2)
    \end{lstlisting}
\end{frame}

%%%%%%%%%%%%%%%%%%%%%%%%%%%%%%%%%%%%%%%%%%%%%%%%%%%%%%%%%%%%%%%%%%%%%%%%%%%%%%%%%%%
\begin{frame}[fragile]\frametitle{NumPy Variables are References}

    NumPy arrays are mutable, so NumPy variables are \textit{references}.
    \\~\\
    Try these commands, then look at \lstinline|x| and \lstinline|y|:
    \begin{lstlisting}
>>> x = np.arange(5)
>>> y = x
>>> x[0] = 5

#     Changing x will change y
    \end{lstlisting}
\end{frame}

%%%%%%%%%%%%%%%%%%%%%%%%%%%%%%%%%%%%%%%%%%%%%%%%%%%%%%%%%%%%%%%%%%%%%%%%%%%%%%%%%%%
\begin{frame}[fragile]\frametitle{Copying NumPy Arrays}

    \textit{Deep Copy}: Duplicate the array in memory
    \begin{lstlisting}
>>> x = np.arange(10)
>>> y = np.copy(x)
>>> x[0] = 5
>>> x
>>> y
    \end{lstlisting}
\end{frame}

%%%%%%%%%%%%%%%%%%%%%%%%%%%%%%%%%%%%%%%%%%%%%%%%%%%%%%%%%%%%%%%%%%%%%%%%%%%%%%%%%%%%
%\begin{frame}[fragile]\frametitle{NumPy Views}
%
%    \textit{View}: Same Data, different properties\\
%    (i.e. a ``different view'' of the data)
%    \\~\\
%    Try these commands, then look at \lstinline|x| and \lstinline|y|:
%    \begin{lstlisting}
%>>> x = np.arange(12)
%>>> y = x[:]    # y = x.view() also works
%>>> y.shape = 3, 4
%>>> x[0] = 12
%    \end{lstlisting}
%    Note: \lstinline|y = x[:]| copies lists, but creates NumPy \textit{views}!
%    \\~\\
%    Subarrays are also views:
%    \begin{lstlisting}
%>>> x = np.arange(5)
%>>> y = x[:3]
%>>> x[0]
%    \end{lstlisting}
%\end{frame}



%%%%%%%%%%%%%%%%%%%%%%%%%%%%%%%%%%%%%%%%%%%%%%%%%%%%%%%%%%%%%%%%%%%%%%%%%%%%%%%%%%%%
%\begin{frame}[fragile]\frametitle{SciPy NetCDF Support}
%
%    SciPy provides some NetCDF support (PuPyNeRe):
%    \begin{lstlisting}
%>>> import scipy.io as sio
%>>> x = np.arange(5)
%>>> f = sio.netcdf.netcdf_file('out.nc','w')
%>>> f.createDimension('xd',5)
%>>> x_var = f.createVariable('x','d',('xd',))
%>>> x_var[:] = x
%>>> f.close()
%    \end{lstlisting}
%    Also see: \lstinline|netcdf4-python|
%\end{frame}

%%%%%%%%%%%%%%%%%%%%%%%%%%%%%%%%%%%%%%%%%%%%%%%%%%%%%%%%%%%%%%%%%%%%%%%%%%%%%%%%%%%
\begin{frame}[fragile]\frametitle{Masking}

    Filtering with logical operators:
    \begin{lstlisting}
>>> x = np.random.rand(3,4)
>>> x > 0.5
>>> x[(x>0.5)]

#     This can be useful, but you lose the shape!
    \end{lstlisting}
\end{frame}

%%%%%%%%%%%%%%%%%%%%%%%%%%%%%%%%%%%%%%%%%%%%%%%%%%%%%%%%%%%%%%%%%%%%%%%%%%%%%%%%%%%
\begin{frame}[fragile]\frametitle{Masked Arrays}

    NumPy provides Masked Array support: \lstinline|np.ma|:
    \begin{lstlisting}
>>> x = np.random.rand(3,4)
>>> x_m = np.ma.masked_array(x, x>0.5)
>>> print x_m

#     Support isn't universal, but it's not bad
    \end{lstlisting}
\end{frame}


%%%%%%%%%%%%%%%%%%%%%%%%%%%%%%%%%%%%%%%%%%%%%%%%%%%%%%%%%%%%%%%%%%%%%%%%%%%%%%%%%%%
\begin{frame}[fragile]\frametitle{SciPy: a toolbox for numerics}

 \begin{itemize}
\item    \href{http://www.scipy.org}{SciPy} is open-source software for
  mathematics, science, and engineering. \emph{[\ldots]} The SciPy
  library provides many user-friendly and efficient numerical routines
  such as routines for numerical integration and optimization.

\item    One of its main aim is to provide a reimplementation of the
  MATLAB toolboxes.

\item     \emph{Use it if:} you long for MATLAB toolbox features.


  \item Tutorial: {\scriptsize\url{http://docs.scipy.org/doc/scipy/reference/tutorial/index.html}}
  \item Examples: {\scriptsize\url{http://nbviewer.ipython.org/github/jrjohansson/scientific-python-lectures/blob/master/Lecture-3-Scipy.ipynb}}
  \end{itemize}
\end{frame}

%%%%%%%%%%%%%%%%%%%%%%%%%%%%%%%%%%%%%%%%%%%%%%%%%%%%%%%%%%%%%%%%%%%%%%%%%%%%%%%%%%%
\begin{frame}[fragile]
    \frametitle{SciPy}

    SciPy provides lots of useful science tools:
    \begin{itemize}
        \item \lstinline|scipy.interpolate|\\
            Grid interpolation tools
        \item \lstinline|scipy.stats|, \lstinline|scipy.random|\\
            Statistical analysis
        \item \lstinline|scipy.signal|\\
            Filtering, Signal Processing
    \end{itemize}
    and many more
\end{frame}

%%%%%%%%%%%%%%%%%%%%%%%%%%%%%%%%%%%%%%%%%%%%%%%%%%%%%%%%%%%%%%%%%%%%%%%%%%%%%%%%%%%
\begin{frame}[fragile]\frametitle{I/O with SciPy}
    
    NumPy has a unique binary data format:
    \begin{lstlisting}
>>> np.save('mydata.npy', data)
>>> data = np.load('mydata.npy')

# Several I/O routines are provided by SciPy (scientific python).

#   Matlab:
>>> import scipy.io as sio
>>> data = sio.loadmat('mydata.mat')
>>> sio.savemat('mydata.mat', {'var': data})
    \end{lstlisting}
\end{frame}
