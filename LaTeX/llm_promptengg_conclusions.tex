%%%%%%%%%%%%%%%%%%%%%%%%%%%%%%%%%%%%%%%%%%%%%%%%%%%%%%%%%%%%%%%%%%%%%%%%%%%%%%%%%%
\begin{frame}[fragile]\frametitle{}
\begin{center}
{\Large Conclusions}
\end{center}
\end{frame}

%%%%%%%%%%%%%%%%%%%%%%%%%%%%%%%%%%%%%%%%%%%%%%%%%%%%%%%%%%%
\begin{frame}[fragile]\frametitle{Progression}

Models for prediction:

\begin{itemize}
\item On data, derive features, put statistical techniques like regression. One model per task. That's Machine Learning.
\item Feed raw data, employ neural networks. One model per task. That's Deep Learning.
\item Use Text data, get embeddings, use ML/DL, say for classification. One model per task. That's Natural Language Processing.
\item Train neural network on large corpus, store weights and architecture, then add final layers for say classification on custom data+labels. That's Pretrained model. One model, many tasks.
\item Train Large Language Model, just supply instructions on what to do, works. One model many tasks. Zero-shot, few-shots.
\end{itemize}

{\tiny (More info at SaaS LLM https://medium.com/google-developer-experts/saasgpt-84ba80265d0f)}

\end{frame}


%%%%%%%%%%%%%%%%%%%%%%%%%%%%%%%%%%%%%%%%%%%%%%%%%%%%%%%%%%%
\begin{frame}[fragile]\frametitle{New Programming Language?}

\begin{center}
\includegraphics[width=0.8\linewidth,keepaspectratio]{promptengg1}

{\tiny (Ref: Prompt Engineering Sudalai Rajkumar)}

\end{center}				

\end{frame}

%%%%%%%%%%%%%%%%%%%%%%%%%%%%%%%%%%%%%%%%%%%%%%%%%%%%%%%%%%%
\begin{frame}[fragile]\frametitle{Summary of Prompt Engineering}

\begin{itemize}
  \item \textbf{Definition of Prompts}
    \begin{itemize}
      \item Prompts are initial text inputs provided to a model.
      \item Used by the model to generate responses or accomplish tasks.
    \end{itemize}

  \item \textbf{Role of Prompts}
    \begin{itemize}
      \item Sets of instructions for AI or chatbots (e.g., ChatGPT).
      \item Applied in various tasks, including summarization, arithmetic problem-solving, and question-answering.
    \end{itemize}

  \item \textbf{Objective of Prompt Engineering}
    \begin{itemize}
      \item Goal: Refine prompts to enhance model accuracy and relevance in outputs.
      \item Central to improving the performance of language models.
    \end{itemize}

  \item \textbf{Prevalent Prompt Types}
    \begin{itemize}
      \item Various prompt types exist, with a focus on two widely used methodologies:
        \begin{itemize}
          \item \textbf{Zero-shot prompting}
          \item \textbf{Few-shot prompting}
        \end{itemize}
    \end{itemize}
\end{itemize}

\end{frame}


%%%%%%%%%%%%%%%%%%%%%%%%%%%%%%%%%%%%%%%%%%%%%%%%%%%%%%%%%%%
\begin{frame}[fragile]\frametitle{ChatGPT Ultimate Prompting Guide}

\begin{itemize}
\item Tone: Specify the desired tone (e.g., formal, casual, informative, persuasive).
\item Format: Define the format or structure (e.g., essay, bullet points, outline, dialogue). 
\item Act as: Indicate a role or perspective to adopt (e.g., expert, critic, enthusiast). 
\item Objective: State the goal or purpose of the response (e.g., inform, persuade, entertain). 
\item Context: Provide background information, data, or context for accurate content generation. 
\item Scope: Define the scope or range of the topic.
\item Keywords: List important keywords or phrases to be included.
\item Limitations: Specify constraints, such as word or character count.
\item Examples: Provide examples of desired style, structure, or content.
\item Deadline: Mention deadlines or time frames for time-sensitive responses. 
\end{itemize}	 


{\tiny (Ref: LinkedIn post by Generative AI, Twitter by Aadit Sheth, Source : Reddit)}
			

\end{frame}


%%%%%%%%%%%%%%%%%%%%%%%%%%%%%%%%%%%%%%%%%%%%%%%%%%%%%%%%%%%
\begin{frame}[fragile]\frametitle{ChatGPT Ultimate Prompting Guide}

\begin{itemize}
\item Audience: Specify the target audience for tailored content. 
\item Language: Indicate the language for the response, if different from the prompt. 
\item Citations: Request inclusion of citations or sources to support information. 
\item Points of view: Ask the Al to consider multiple perspectives or opinions. 
\item Counter arguments: Request addressing potential counterarguments. 
\item Terminology: Specify industry-specific or technical terms to use or avoid.
\item Analogies: Ask the Al to use analogies or examples to clarify concepts. 
\item Quotes: Request inclusion of relevant quotes or statements from experts. 
\item Statistics: Encourage the use of statistics or data to support claims. 
\item Visual elements: Inquire about including charts, graphs, or images. 
\item Call to action: Request a clear call to action or next steps. 
\item Sensitivity: Mention sensitive topics or issues to be handled with care or avoided. 
\end{itemize}	 


{\tiny (Ref: LinkedIn post by Generative AI, Twitter by Aadit Sheth, Source : Reddit)}
			

\end{frame}

%%%%%%%%%%%%%%%%%%%%%%%%%%%%%%%%%%%%%%%%%%%%%%%%%%%%%%%%%%%
\begin{frame}[fragile]\frametitle{Interaction Guidelines: Avoid Misuses}

\begin{itemize}
\item Factual Accuracy: Interactions must be free from factual inaccuracies that can be challenged by social media or journalists.
\item Negative Debates: Avoid discussing topics that fuel negative or concerning online debates, such as AI sentience, AI in education, AI-driven job displacements, and politically divisive issues.
\item Minors' Involvement: Do not include use cases specifically targeting or involving individuals under 18 years old.
\item Sensitivity and Misinformation: Prevent the inclusion of sensitive, misleading, or hazardous responses.
\item Search and Google Assistant: Interactions that require basic, straightforward answers are better suited for Search or Google Assistant.
\item Financial/Legal/Medical Advice: Refrain from providing advice related to financial matters, legal issues, or medical concerns.
\item Brand Names and Trademarks: Avoid mentioning specific brand names, trademarks, or public figures (except historical figures).
\item No Reviews or Tweets: Do not request reviews of restaurants, businesses, or tweets to minimize the risk of associating with bots.
\item Avoid Personification: Refrain from personifying the product or brand and from encouraging users to address Bard by name.
\end{itemize}	 
\end{frame}


%%%%%%%%%%%%%%%%%%%%%%%%%%%%%%%%%%%%%%%%%%%%%%%%%%%%%%%%%%%
\begin{frame}[fragile]\frametitle{Limitations}


Boie is a real company, the product name is not real. So, see what you get \ldots

\begin{lstlisting}
prompt = f"""
Tell me about AeroGlide UltraSlim Smart Toothbrush by Boie
"""
response = get_completion(prompt)
print(response)
\end{lstlisting}	 
		
\end{frame}

%%%%%%%%%%%%%%%%%%%%%%%%%%%%%%%%%%%%%%%%%%%%%%%%%%%%%%%%%%%%%%%%%%%%%%%%%%%%%%%%%%
\begin{frame}[fragile]\frametitle{}
\begin{center}
{\Large What Next?}
\end{center}
\end{frame}

%%%%%%%%%%%%%%%%%%%%%%%%%%%%%%%%%%%%%%%%%%%%%%%%%%%%%%%%%%%
\begin{frame}[fragile]\frametitle{The Career of the Future}

\begin{center}
\includegraphics[width=\linewidth,keepaspectratio]{promptengg62}

{\tiny (Ref: The Complete Prompt Engineering for AI Bootcamp (2023))}

\end{center}				
\end{frame}

%%%%%%%%%%%%%%%%%%%%%%%%%%%%%%%%%%%%%%%%%%%%%%%%%%%%%%%%%%%
\begin{frame}[fragile]\frametitle{New Roles?}

Coming up with good prompt is a combination of art and science

\begin{center}
\includegraphics[width=0.8\linewidth,keepaspectratio]{promptengg5}

{\tiny (Ref: Prompt Engineering Sudalai Rajkumar)}

\end{center}
\end{frame}



%%%%%%%%%%%%%%%%%%%%%%%%%%%%%%%%%%%%%%%%%%%%%%%%%%%%%%%%%%%

\begin{frame}[fragile]\frametitle{Read on to learn how to engineer good prompts!}

\begin{itemize}
\item Shin, T., Razeghi, Y., Logan IV, R. L., Wallace, E., \& Singh, S. (2020). AutoPrompt: Eliciting Knowledge from Language Models with Automatically Generated Prompts. Proceedings of the 2020 Conference on Empirical Methods in Natural Language Processing (EMNLP). https://doi.org/10.18653/v1/2020.emnlp-main.346 
\item Kojima, T., Gu, S. S., Reid, M., Matsuo, Y., \& Iwasawa, Y. (2022). Large Language Models are Zero-Shot Reasoners. 
\item Liu, P., Yuan, W., Fu, J., Jiang, Z., Hayashi, H., \& Neubig, G. (2022). Pre-train, Prompt, and Predict: A Systematic Survey of Prompting Methods in Natural Language Processing. ACM Computing Surveys. https://doi.org/10.1145/3560815 
\item Brown, T. B., Mann, B., Ryder, N., Subbiah, M., Kaplan, J., Dhariwal, P., Neelakantan, A., Shyam, P., Sastry, G., Askell, A., Agarwal, S., Herbert-Voss, A., Krueger, G., Henighan, T., Child, R., Ramesh, A., Ziegler, D. M., Wu, J., Winter, C., … Amodei, D. (2020). Language Models are Few-Shot Learners. 
\item Zhao, T. Z., Wallace, E., Feng, S., Klein, D., \& Singh, S. (2021). Calibrate Before Use: Improving Few-Shot Performance of Language Models.
\end{itemize}
\end{frame}



%%%%%%%%%%%%%%%%%%%%%%%%%%%%%%%%%%%%%%%%%%%%%%%%%%%%%%%%%%%
\begin{frame}[fragile]\frametitle{My Sketchnote}

\begin{center}
\includegraphics[width=0.45\linewidth,keepaspectratio]{PromptEng_Sketchnote_Medium}

{\tiny (Ref: https://medium.com/technology-hits/prompting-is-all-you-need-5dddb82bd022)}
\end{center}		

\end{frame}

%%%%%%%%%%%%%%%%%%%%%%%%%%%%%%%%%%%%%%%%%%%%%%%%%%%%%%%%%%%

\begin{frame}[fragile]\frametitle{Take Aways}

Prompt Engineering is an Iterative Process:

\begin{itemize}
\item Try something
\item Analyze where the results do not match the expectations
\item Clarify instructions, gives examples, specify output format, specify constraints, etc
\item Test on a batch of known results.
\end{itemize}
\end{frame}

%%%%%%%%%%%%%%%%%%%%%%%%%%%%%%%%%%%%%%%%%%%%%%%%%%%%%%%%%%%
\begin{frame}[fragile]\frametitle{Quality of Ideas}

\begin{center}
\includegraphics[width=\linewidth,keepaspectratio]{promptengg63}

{\tiny (Ref: The Complete Prompt Engineering for AI Bootcamp (2023))}

\end{center}				
\end{frame}


%%%%%%%%%%%%%%%%%%%%%%%%%%%%%%%%%%%%%%%%%%%%%%%%%%%%%%%%%%%%%%%%%%%%%%%%%%%%%%%%%%
\begin{frame}[fragile]\frametitle{Resources}
\begin{itemize}
\item Prompt Engineering Guide https://github.com/dair-ai/Prompt-Engineering-Guide
\item Awesome ChatGPT Prompts https://github.com/f/awesome-chatgpt-prompts/
\item ChatGPT Prompt Engineering for Developers - Deep Learning AI
\item Learn Prompting https://learnprompting.org/docs/intro
\item Types of Prompts with Practical examples - Dr. Naveed Siddiqui 
\item AI Prompt Database https://justunderstandingdata.notion.site/d98dcc9a6736471584d53cc8b2a5c30d?v=6e6a5c0a824942bdaef4611511f6c91d
\end{itemize}	 
\end{frame}

