%%%%%%%%%%%%%%%%%%%%%%%%%%%%%%%%%%%%%%%%%%%%%%%%%%%%%%%%%%%
\begin{frame}[fragile]\frametitle{}
\begin{center}
{\Large Assignments in Agriculture}
\end{center}
\end{frame}
 
%%%%%%%%%%%%%%%%%%%%%%%%%%%%%%%%%%%%%%%%%%%%%%%%%%%%%%%%%%%
% Slide 1: Crop Yield Prediction using Linear Regression
%%%%%%%%%%%%%%%%%%%%%%%%%%%%%%%%%%%%%%%%%%%%%%%%%%%%%%%%%%%
\begin{frame}[fragile]\frametitle{Crop Yield Prediction using Linear Regression}
    \begin{itemize}
        \item \textbf{Dataset}: Crop Yield Prediction Dataset
        \item \textbf{Source}: \href{https://www.kaggle.com/datasets/sagnik1511/crop-yield-prediction}{Kaggle}
        \item \textbf{Description}: Predict crop yields based on environmental factors such as temperature, humidity, and rainfall using linear regression.
    \end{itemize}
    \begin{lstlisting}[language=Python]
import pandas as pd
from sklearn.model_selection import train_test_split
from sklearn.linear_model import LinearRegression

data = pd.read_csv('crop_yield.csv')
X = data[['temperature', 'humidity', 'rainfall']]
y = data['yield']

X_train, X_test, y_train, y_test = train_test_split(X, y, test_size=0.2)
model = LinearRegression()
model.fit(X_train, y_train)
predictions = model.predict(X_test)
    \end{lstlisting}
\end{frame}

%%%%%%%%%%%%%%%%%%%%%%%%%%%%%%%%%%%%%%%%%%%%%%%%%%%%%%%%%%%
% Slide 2: Disease Detection in Crops using Decision Trees
%%%%%%%%%%%%%%%%%%%%%%%%%%%%%%%%%%%%%%%%%%%%%%%%%%%%%%%%%%%
\begin{frame}[fragile]\frametitle{Disease Detection in Crops using Decision Trees}
    \begin{itemize}
        \item \textbf{Dataset}: Plant Village Dataset
        \item \textbf{Source}: \href{https://www.kaggle.com/datasets/emmarex/plantdisease}{Kaggle}
        \item \textbf{Description}: Use decision trees to classify crop diseases based on features extracted from images of leaves.
    \end{itemize}
    \begin{lstlisting}[language=Python]
from sklearn.tree import DecisionTreeClassifier
from sklearn.model_selection import train_test_split
import pandas as pd

data = pd.read_csv('plant_disease.csv')
X = data.drop('disease', axis=1)
y = data['disease']

X_train, X_test, y_train, y_test = train_test_split(X, y, test_size=0.2)
model = DecisionTreeClassifier()
model.fit(X_train, y_train)
accuracy = model.score(X_test, y_test)
    \end{lstlisting}
\end{frame}

%%%%%%%%%%%%%%%%%%%%%%%%%%%%%%%%%%%%%%%%%%%%%%%%%%%%%%%%%%%
% Slide 3: Weed Detection using Convolutional Neural Networks (CNN)
%%%%%%%%%%%%%%%%%%%%%%%%%%%%%%%%%%%%%%%%%%%%%%%%%%%%%%%%%%%
\begin{frame}[fragile]\frametitle{Weed Detection using Convolutional Neural Networks (CNN)}
    \begin{itemize}
        \item \textbf{Dataset}: Weeds Dataset
        \item \textbf{Source}: \href{https://www.kaggle.com/datasets/abhishek/weed-detection-dataset}{Kaggle}
        \item \textbf{Description}: Implement a CNN to detect weeds in crop fields by training on images of weeds versus crops.
    \end{itemize}
    \begin{lstlisting}[language=Python]
from keras.models import Sequential
from keras.layers import Conv2D, MaxPooling2D, Flatten, Dense

model = Sequential()
model.add(Conv2D(32, (3,3), activation='relu', input_shape=(64,64,3)))
model.add(MaxPooling2D(pool_size=(2,2)))

model.add(Flatten())
model.add(Dense(units=128, activation='relu'))
model.add(Dense(units=1, activation='sigmoid'))

model.compile(optimizer='adam', loss='binary_crossentropy', metrics=['accuracy'])
    \end{lstlisting}
\end{frame}

%%%%%%%%%%%%%%%%%%%%%%%%%%%%%%%%%%%%%%%%%%%%%%%%%%%%%%%%%%%
% Slide 4: Soil Quality Prediction using Random Forests
%%%%%%%%%%%%%%%%%%%%%%%%%%%%%%%%%%%%%%%%%%%%%%%%%%%%%%%%%%%
\begin{frame}[fragile]\frametitle{Soil Quality Prediction using Random Forests}
    \begin{itemize}
        \item \textbf{Dataset}: Soil Quality Dataset
        \item \textbf{Source}: \href{https://archive.ics.uci.edu/ml/datasets/Soil+Quality}{UCI Machine Learning Repository}
        \item \textbf{Description}: Use random forests to predict soil quality based on various chemical and physical properties of the soil.
    \end{itemize}
    \begin{lstlisting}[language=Python]
import pandas as pd
from sklearn.ensemble import RandomForestClassifier
from sklearn.model_selection import train_test_split

data = pd.read_csv('soil_quality.csv')
X = data.drop('quality', axis=1)
y = data['quality']

X_train, X_test, y_train, y_test = train_test_split(X, y, test_size=0.3)
model = RandomForestClassifier()
model.fit(X_train, y_train)

accuracy = model.score(X_test, y_test)
    \end{lstlisting}
\end{frame}

%%%%%%%%%%%%%%%%%%%%%%%%%%%%%%%%%%%%%%%%%%%%%%%%%%%%%%%%%%%
% Slide 5: Irrigation Scheduling using Support Vector Machines (SVM)
%%%%%%%%%%%%%%%%%%%%%%%%%%%%%%%%%%%%%%%%%%%%%%%%%%%%%%%%%%%
\begin{frame}[fragile]\frametitle{Irrigation Scheduling using Support Vector Machines (SVM)}
    \begin{itemize}
        \item \textbf{Dataset}: Irrigation Scheduling Dataset
        \item \textbf{Source}: \href{https://www.kaggle.com/datasets/krishnaik06/irrigation-scheduling-dataset}{Kaggle}
        \item \textbf{Description}: Implement SVM to determine optimal irrigation schedules based on weather and soil moisture data.
    \end{itemize}
    \begin{lstlisting}[language=Python]
from sklearn.svm import SVC
from sklearn.model_selection import train_test_split
import pandas as pd

data = pd.read_csv('irrigation_data.csv')
X = data.drop('schedule', axis=1)
y = data['schedule']

X_train, X_test, y_train, y_test = train_test_split(X, y, test_size=0.2)
model = SVC()
model.fit(X_train, y_train)

accuracy = model.score(X_test, y_test)
    \end{lstlisting}
\end{frame}

%%%%%%%%%%%%%%%%%%%%%%%%%%%%%%%%%%%%%%%%%%%%%%%%%%%%%%%%%%%
% Slide 6: Crop Disease Prediction using K-Nearest Neighbors (KNN)
%%%%%%%%%%%%%%%%%%%%%%%%%%%%%%%%%%%%%%%%%%%%%%%%%%%%%%%%%%%
\begin{frame}[fragile]\frametitle{Crop Disease Prediction using K-Nearest Neighbors (KNN)}
    \begin{itemize}
        \item \textbf{Dataset}: Crop Disease Dataset
        \item \textbf{Source}: \href{https://www.kaggle.com/datasets/ashishpatel26/crop-disease-prediction-dataset}{Kaggle}
        \item \textbf{Description}: Use KNN to classify different diseases affecting crops based on features like leaf color and texture.
    \end{itemize}
    \begin{lstlisting}[language=Python]
from sklearn.neighbors import KNeighborsClassifier
from sklearn.model_selection import train_test_split
import pandas as pd

data = pd.read_csv('crop_disease_data.csv')
X = data.drop('disease', axis=1)
y = data['disease']

X_train, X_test, y_train, y_test = train_test_split(X, y, test_size=0.25)
model = KNeighborsClassifier(n_neighbors=5)
model.fit(X_train, y_train)

accuracy = model.score(X_test, y_test)
    \end{lstlisting}
\end{frame}

%%%%%%%%%%%%%%%%%%%%%%%%%%%%%%%%%%%%%%%%%%%%%%%%%%%%%%%%%%%
% Slide 7: Yield Prediction using Gradient Boosting Machines (GBM)
%%%%%%%%%%%%%%%%%%%%%%%%%%%%%%%%%%%%%%%%%%%%%%%%%%%%%%%%%%%
\begin{frame}[fragile]\frametitle{Yield Prediction using Gradient Boosting Machines (GBM)}
    \begin{itemize}
        \item \textbf{Dataset}: US Crop Yield Data
        \item \textbf{Source}: \href{https://www.nass.usda.gov/}{USDA}
        \item \textbf{Description}: Apply gradient boosting to predict crop yields based on historical yield data and environmental factors.
    \end{itemize}
    \begin{lstlisting}[language=Python]
import pandas as pd
from sklearn.ensemble import GradientBoostingRegressor
from sklearn.model_selection import train_test_split

data = pd.read_csv('us_crop_yield.csv')
X = data.drop('yield', axis=1)
y = data['yield']

X_train, X_test, y_train, y_test = train_test_split(X, y, test_size=0.2)
model = GradientBoostingRegressor()
model.fit(X_train, y_train)

predictions = model.predict(X_test)
    \end{lstlisting}
\end{frame}

%%%%%%%%%%%%%%%%%%%%%%%%%%%%%%%%%%%%%%%%%%%%%%%%%%%%%%%%%%%
% Slide 8: Pest Detection using Image Classification with CNNs
%%%%%%%%%%%%%%%%%%%%%%%%%%%%%%%%%%%%%%%%%%%%%%%%%%%%%%%%%%%
\begin{frame}[fragile]\frametitle{Pest Detection using Image Classification with CNNs}
    \begin{itemize}
        \item \textbf{Dataset}: Pest Image Dataset
        \item \textbf{Source}: \href{https://www.kaggle.com/datasets/sarfaraz/pest-detection-dataset}{Kaggle}
        \item \textbf{Description}: Train a CNN for pest detection in crops by classifying images of healthy versus infested plants.
    \end{itemize}
    \begin{lstlisting}[language=Python]
from keras.models import Sequential
from keras.layers import Conv2D, MaxPooling2D, Flatten, Dense

model = Sequential()
model.add(Conv2D(32,(3,3), activation='relu', input_shape=(128,128,3)))
model.add(MaxPooling2D(pool_size=(2,2)))
model.add(Flatten())
model.add(Dense(units=128 , activation='relu'))
model.add(Dense(units=10 , activation='softmax'))

model.compile(optimizer='adam', loss='categorical_crossentropy', metrics=['accuracy'])
    \end{lstlisting}
\end{frame}

%%%%%%%%%%%%%%%%%%%%%%%%%%%%%%%%%%%%%%%%%%%%%%%%%%%%%%%%%%%
% Slide 9: Precision Agriculture using Clustering Techniques (K-Means)
%%%%%%%%%%%%%%%%%%%%%%%%%%%%%%%%%%%%%%%%%%%%%%%%%%%%%%%%%%%
\begin{frame}[fragile]\frametitle{Precision Agriculture using Clustering Techniques (K-Means)}
    \begin{itemize}
        \item \textbf{Dataset}: Precision Agriculture Dataset
        \item \textbf{Source}: \href{https://www.kaggle.com/datasets/sagnik1511/precision-agriculture-data-set}{Kaggle}
        \item \textbf{Description}: Use K-means clustering to segment agricultural land into different zones for precision farming practices.
    \end{itemize}
    \begin{lstlisting}[language=Python]
import pandas as pd
from sklearn.cluster import KMeans

data = pd.read_csv('precision_agriculture.csv')
X = data[['latitude', 'longitude']]

kmeans = KMeans(n_clusters=3) # Number of clusters can be adjusted.
kmeans.fit(X)

data['cluster'] = kmeans.labels_
    \end{lstlisting}
\end{frame}

%%%%%%%%%%%%%%%%%%%%%%%%%%%%%%%%%%%%%%%%%%%%%%%%%%%%%%%%%%%
% Slide 10: Weather Forecasting for Agriculture using LSTM Networks
%%%%%%%%%%%%%%%%%%%%%%%%%%%%%%%%%%%%%%%%%%%%%%%%%%%%%%%%%%%
\begin{frame}[fragile]\frametitle{Weather Forecasting for Agriculture using LSTM Networks}
    \begin{itemize}
        \item \textbf{Dataset}: Weather Data
        \item \textbf{Source}: \href{https://www.ncdc.noaa.gov/cdo-web/}{NOAA}
        \item \textbf{Description}: Implement LSTM networks to forecast weather conditions that affect agricultural practices like planting and harvesting.
    \end{itemize}
    \begin{lstlisting}[language=Python]
import numpy as np
import pandas as pd
from keras.models import Sequential
from keras.layers import LSTM,Dense

data = pd.read_csv('weather_data.csv')
X,y=data.iloc[:,:-1],data.iloc[:,-1]

model=Sequential()
model.add(LSTM(50,input_shape=(X.shape[1],X.shape[2])))
model.add(Dense(1))
model.compile(loss='mean_squared_error', optimizer='adam')
    \end{lstlisting}
\end{frame}

