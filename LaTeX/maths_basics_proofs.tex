
%%%%%%%%%%%%%%%%%%%%%%%%%%%%%%%%%%%%%%%%%%%%%%%%%%%%%%%%%%%%%%%%%%%%%%%%%%%%%%%%%%
\begin{frame}[fragile]\frametitle{}
\begin{center}
{\Large Proofs}

\tiny{(Ref: CS 207 Discrete Mathematics – 2012-2013, IITB, 
Nutan Limaye
 https://www.cse.iitb.ac.in/~nutan/courses/cs207-12/notes/)}
\end{center}
\end{frame}



\newtheorem{axiom}{Axiom}[theorem]
\newtheorem{conj}{Conjecture}[theorem]


%%%%
%%%% colors
%%%%
\newcommand{\TealBlue}[1]{\textcolor{TealBlue}{#1}}
\newcommand{\Peach}[1]{\textcolor{Peach}{#1}}
\newcommand{\Cyan}[1]{\textcolor{cyan}{#1}}
\newcommand{\Red}[1]{\textcolor{red}{#1}}
\newcommand{\BrickRed}[1]{\textcolor{BrickRed}{#1}}
\newcommand{\Blue}[1]{\textcolor{blue}{#1}}
\newcommand{\Green}[1]{\textcolor{green}{#1}}
\newcommand{\Black}[1]{\textcolor{black}{#1}}
\newcommand{\White}[1]{\textcolor{white}{#1}}
\newcommand{\Blueit}[1]{\textcolor{Blue}{\emph{#1}}}
\newcommand{\Magentait}[1]{\textcolor{Magenta}{\emph{#1}}}
\newcommand{\Orangeit}[1]{\textcolor{Orange}{\emph{#1}}}
\newcommand{\Greenit}[1]{\textcolor{Green}{\emph{#1}}}
\newcommand{\Cyanit}[1]{\textcolor{cyan}{\emph{#1}}}


%
%  Macros
%

\newcommand{\RingNat}{{{\mathbb{N}}}}
\newcommand{\RingInt}{{{\mathbb{Z}}}}
\newcommand{\RingReal}{{{\mathbb{R}}}}
\newcommand{\RingRat}{{{\mathbb{Q}}}}
\newcommand{\RingPosInt}{{{\mathbb{Z^+}}}}
\newcommand{\myand}{\wedge}
\newcommand{\myor}{\vee}
\newcommand{\mynot}{\neg}


%%%%%%%%%%%%%%%%%%%%%%%%%%%%%%%%%%%%%%%%%%%%%%%%%%%%%%%%%%%%%%%%%%%%%%%%%%%%%%%%
 \begin{frame}[fragile]\frametitle{What is a proposition?}
A statement that is either true or false.

\begin{itemize}
\item $2+2 = 4$, every odd number is a prime, there are no even primes other than 2;

\item $\forall a,b \in \RingNat, \exists c \in \RingNat: a^2 + b^2 = c$;



$\forall:$ for all, \\
$\exists:$ there exists, \\
$\in, \notin:$ contained in, and not contained in\\

$\RingNat:$ the set of natural numbers,\\
 $\RingInt:$ the set of integers,\\ 
 $\RingRat:$ the set of rationals,\\ 
 $\RingPosInt:$ the set of positive integers,\\
 $\RingReal:$ the set of reals\\
 


\item $\forall a,b \in \RingNat, \exists c \in \RingNat: a^2 - b^2 = c$;


\item $\forall a,b \in \RingNat, \exists c \in \RingInt: a^2 - b^2 = c$;
\end{itemize}

It is not always easy to tell whether a proposition is true or false. 

\end{frame}

%%%%%%%%%%%%%%%%%%%%%%%%%%%%%%%%%%%%%%%%%%%%%%%%%%%%%%%%%%%%%%%%%%%%%%%%%%%%%%%%
 \begin{frame}[fragile]\frametitle{Theorems and proofs}
\begin{theorem}
If $0 \leq x \leq 2,$ then $-x^3+4x+1 > 0$
\end{theorem}

*(scratchpad)*

\begin{proof}
As $-x^3+4x = x(4-x^2)$, which is in fact $x(2-x)(2+x)$, the quantity is 
\textbf{positive} 
non-negative for $0\leq x \leq 2$.
Adding $1$ to a non-negative quantity makes it positive. Therefore, the above theorem.
\end{proof}
\end{frame}

%%%%%%%%%%%%%%%%%%%%%%%%%%%%%%%%%%%%%%%%%%%%%%%%%%%%%%%%%%%%%%%%%%%%%%%%%%%%%%%%
 \begin{frame}[fragile]
\frametitle{Theorems and Proofs}

\begin{tabular}{ll}
Given: & a number $n \in \RingNat$\\
Check: & Is $n$ prime?\\
& \\
\end{tabular}

Why is this algorithm correct?\\

Is there a number $n \in \RingNat$ s.t \\
$\forall i: i \in \{2,3,\ldots, \sqrt{n}\}$ $i \nmid n$,\\
but $\exists j > \sqrt{n}$ s.t. $j | n$?\\
\vspace{0.1in}
Is there a composite $n \in \RingNat$ s.t. all its prime factors are greater than $\sqrt{n}$?\\
\vspace{0.1in}
%
\begin{lstlisting}
\FOR{$i=2$ to $\sqrt{n}$}
\IF{$i | n$}
\STATE output ``no''
\ENDIF
\ENDFOR
\end{lstlisting}

\end{frame}
%%%%%%%%%%%%%%%%%%%%%%%%%%%%%%%%%%%%%%%%%%%%%%%%%%%%%%%%%%%%%%%%%%%%%%%%%%%%%%%%
 \begin{frame}[fragile]
\frametitle{Theorems and Proofs}

\begin{theorem}
If $n$ is a composite integer, then $n$ has a prime divisor less than or equal to $\sqrt{n}$
\end{theorem}
\begin{proof}
As $n$ is a composite, $\exists x,y \in \RingNat, x,y < n:$ $n =xy$. If $x > \sqrt{n}$ and $y > \sqrt{n}$ then
$xy > n$. Therefore, one of $x$ or $y$ is less than or equal to $\sqrt{n}$. Say $x$ is smaller than $\sqrt{n}$. It is either a composite or a prime. If it is a prime, then we are done. Else, it has prime factorization (axiom: unique factorization in $\RingNat$) and again, we are done.
\end{proof}
\end{frame}


%%%%%%%%%%%%%%%%%%%%%%%%%%%%%%%%%%%%%%%%%%%%%%%%%%%%%%%%%%%%%%%%%%%%%%%%%%%%%%%%
 \begin{frame}[fragile]
\frametitle{Axioms}
Euclid in 300BC invented the method of axioms-and-proofs. \\
\vspace*{0.2in}
Using only a handful of axioms called Zermelo-Fraenkel and Choice (ZFC) and a few rules of deductions
the entire mathematics can be deduced!\\
\vspace*{0.2in}
Proving theorems starting from ZFC alone is tedious. 20,000+ lines proof for $2+2=4$\\
\vspace*{0.2in}
We will assume a whole lot of axioms to prove theorems: all familiar facts from high school math.
\end{frame}

%%%%%%%%%%%%%%%%%%%%%%%%%%%%%%%%%%%%%%%%%%%%%%%%%%%%%%%%%%%%%%%%%%%%%%%%%%%%%%%%
 \begin{frame}[fragile]
\frametitle{Class problems}
\begin{itemize}
\item \textcolor{blue}{(CW1.1)} Prove that for any $n \in \RingNat$, $n(n^2-1)(n+2)$ is divisible by $4$. (what about divisible by $8$?)
\item \textcolor{blue}{(CW1.2)} Prove that for any $n \in \RingNat$, $2^n < (n+2)!$ 
\end{itemize}
\end{frame}

 \begin{frame}[fragile]
\frametitle{Bogus proofs}
\begin{theorem}[Bogus]
$1/8 > 1/4$
\end{theorem}
\begin{proof}
\begin{align*}
3 & > 2\\
3 \log_{10} (1/2)& >  2 \log_{10}(1/2)\\
\log_{10} (1/2)^3 & > \log_{10}(1/2)^2\\
(1/2)^3 & > (1/2)^2\\
\end{align*}
\end{proof}
\end{frame}

%%%%%%%%%%%%%%%%%%%%%%%%%%%%%%%%%%%%%%%%%%%%%%%%%%%%%%%%%%%%%%%%%%%%%%%%%%%%%%%%
 \begin{frame}[fragile]
\frametitle{Another bogus proof}
\begin{theorem}
For all non-negative numbers $a,b$ $\frac{a+b}{2} \geq \sqrt{ab}$
\end{theorem}

\begin{proof}
\begin{align*}
\frac{a+b}{2} & \geq^? \sqrt{ab}\\
a+b & \geq^? 2\sqrt{ab}\\
a^2 + 2ab + b^2 & \geq^? 4ab\\
a^2 - 2ab + b^2 & \geq^? 0\\
(a-b)^2 & \geq 0\\
\end{align*}
		
\end{proof}

\end{frame}

%%%%%%%%%%%%%%%%%%%%%%%%%%%%%%%%%%%%%%%%%%%%%%%%%%%%%%%%%%%%%%%%%%%%%%%%%%%%%%%%
 \begin{frame}[fragile]
\frametitle{Proof by contrapositive}
\begin{theorem}
If $r$ is irrational then $\sqrt{r}$ is also irrational.\\
If $\sqrt{r}$ is rational then $r$ is rational.
\end{theorem}

%What is the contrapositive of the above statement? 

\begin{definition}[Contrapozitive]
The contrapositive of ``if $P$ then $Q$'' is ``if $\neg Q$ then $\neg P$''
\end{definition}


\begin{proof}
Suppose $\sqrt{r}$ is rational. Then $\sqrt{r}= p/q$ for $p,q \in \RingInt$. Therefore, $r = p^2/q^2$. 
\end{proof}
\end{frame}

%%%%%%%%%%%%%%%%%%%%%%%%%%%%%%%%%%%%%%%%%%%%%%%%%%%%%%%%%%%%%%%%%%%%%%%%%%%%%%%%
 \begin{frame}[fragile]
\frametitle{Proof by contradiction: $\sqrt{2}$ is irrational.}

\begin{proof}
Suppose not. Then there exists $p,q \in \RingInt$ such that $\sqrt{2}=p/q,$ where $p,q$ do not have any common divisors. Therefore, $2q^2 = p^2$, i.e. $p^2$ is even. \\

\textcolor{blue}{(CW2.1)} If $p^2$ is even, then $p$ is even. 


(why?) \\
Suppose not, i..e $p^2$ is even but $p$ is not. Then $p = 2k +1$ for some integer $k$. $p^2 = (2k+1)^2=
4k^2+4k +1$. As $4(k^2+k)$ is even,  $4k^2+4k +1$ is odd, which is a contradiction.\\



Therefore, $p=2k$ for some $k \in \RingInt$ $\Rightarrow$ $2q^2 = 4k^2$ $\Rightarrow$ $q^2 = 2k^2$ $\Rightarrow$ $q^2$ is even. Therefore, $q$ is even. That is, $p,q$ have a common factor. This leads to a contradiction.  
\end{proof}

\textcolor{blue}{(CW2.2)} Prove that there are infinitely many primes.
\end{frame}

%%%%%%%%%%%%%%%%%%%%%%%%%%%%%%%%%%%%%%%%%%%%%%%%%%%%%%%%%%%%%%%%%%%%%%%%%%%%%%%%
 \begin{frame}[fragile]
\frametitle{Well-ordering principle and Induction}
%The well-ordering principle (which we will assume as one of our axioms) is:
\begin{axiom}[WOP]
Every nonempty set of non-negative integers has a smallest element. 
\end{axiom}


\begin{axiom}[Induction]
Let $P(n)$ be a property of non-negative integers. If 
\begin{enumerate}
\item $P(0)$ is true (Base case)
\item for all $n \geq 0$, $P(n) \Rightarrow P(n+1)$ (Induction step)
\end{enumerate}
then $P(n)$ is true for for all $n \in \RingNat$.
\end{axiom}

\end{frame}


%%%%%%%%%%%%%%%%%%%%%%%%%%%%%%%%%%%%%%%%%%%%%%%%%%%%%%%%%%%%%%%%%%%%%%%%%%%%%%%%
 \begin{frame}[fragile]
\frametitle{Well-ordering principle and Induction}
\begin{axiom}[Strong Induction]
Let $P(n)$ be a property of non-negative integers. If 
\begin{enumerate}
\item $P(0)$ is true (Base case)
\item $[\forall k \in \{0,1, \ldots, n\}: P(k)] \Rightarrow P(n+1)$ (Induction step)
\end{enumerate}
then $P(n)$ is true for for all $n \in \RingNat$.
\end{axiom}
%\begin{theorem}
%The above two axioms are equivalent. 
%\end{theorem}
\end{frame}

%%%%%%%%%%%%%%%%%%%%%%%%%%%%%%%%%%%%%%%%%%%%%%%%%%%%%%%%%%%%%%%%%%%%%%%%%%%%%%%%
 \begin{frame}[fragile]
\frametitle{WOP $\Rightarrow$ Induction}
\begin{theorem}
Well-ordering principle implies Induction
\end{theorem}
\begin{proof}
Let $P(0)$ be true and for each $n \geq 0$, let $P(n) \Rightarrow P(n+1)$. \\
Let us assume for the sake of contradiction that $P(n)$ is not true for all positive integers. \\
Let $C = \{i \mid P(i) \mbox{ is false}\}$. As $C$ is non-empty and non-negative integers $C$ has a smallest element (due to WOP), say $i_0$.\\
Now, $i_0\neq 0$. Also $P(i_0-1)$ is true, as $i_0-1$ is not in $C$. But $P(i_0-1) \Rightarrow P(i_0)$, which is a contradiction.
\end{proof}

\begin{theorem}
WOP $\Leftrightarrow$ Induction $\Leftrightarrow$ Strong Induction \textcolor{red}{[HW]}
\end{theorem}

\end{frame}

%%%%%%%%%%%%%%%%%%%%%%%%%%%%%%%%%%%%%%%%%%%%%%%%%%%%%%%%%%%%%%%%%%%%%%%%%%%%%%%%
 \begin{frame}[fragile]
\frametitle{Using Induction to prove theorems}
\begin{theorem}
$2^n \leq (n+1)!$
\end{theorem}
\begin{proof}
Base case ($n=0$): $2^0 = 1 = 1!$\\

Induction hypothesis: $2^n \leq (n+1)!$. 
\begin{align*}
2^{n+1} & = 2 \cdot 2^n \\
& \leq 2 \cdot (n+1)! \mbox{     (by indiction hypothesis)}\\
& \leq (n+2) \cdot (n+1)! \\
& \leq (n+2)!
\end{align*}
\end{proof}
\end{frame}

%%%%%%%%%%%%%%%%%%%%%%%%%%%%%%%%%%%%%%%%%%%%%%%%%%%%%%%%%%%%%%%%%%%%%%%%%%%%%%%%
 \begin{frame}[fragile]
\frametitle{Using Well-ordering principle to prove theorems}
Here is a slightly non-trivial example. The following equation does not have any solutions over $\RingNat:$\\

$4a^3+2b^3 = c^3$


\begin{proof}
Suppose (for the sake of contradiction) this has a solution over $\RingNat.$ \\
%Let $S$ be a set of solutions.\\
Let $(A,B,C)$ be the solution with the smallest value of $b$ in $S$. \\

(Such an $s$ exists due to WOP.) 

Observe that $C^3$ is even. Therefore, $C$ is even. Say $C = 2\gamma$. \\
Therefore, $4A^3 + 2B^3 = 8\gamma^3,$ i.e. $2A^3 + B^3 = 4\gamma^3$. \\

Now, $B^3$ is even and so is $B$. Say $B = 2\beta$. $\therefore 2A^3 + 8\beta^3 = 4\gamma^3$.\\

And, now we can repeat the argument with respect to $A$. \\
Therefore, if $(A,B,C)$ is a solution then so is $(\alpha, \beta, \gamma)$.\\

But $\beta < B$, which is a contradiction. 

\end{proof}


\end{frame}


%%%%%%%%%%%%%%%%%%%%%%%%%%%%%%%%%%%%%%%%%%%%%%%%%%%%%%%%%%%%%%%%%%%%%%%%%%%%%%%%
 \begin{frame}[fragile]
\frametitle{Using Well-ordering principle to prove theorems}
It is not always as easy to prove such theorems.
\begin{conj}[Euler, 1769]
There are no positive integer solutions over $\RingInt$ to the
equation:
$$a^4 + b^4 + c^4 = d^4$$
\end{conj}

Integer values for $a, b, c, d$ that do satisfy this equation were first discovered in
1986. \\
It took more two hundred years to prove it.
\end{frame}

%%%%%%%%%%%%%%%%%%%%%%%%%%%%%%%%%%%%%%%%%%%%%%%%%%%%%%%%%%%%%%%%%%%%%%%%%%%%%%%%
 \begin{frame}[fragile]
\frametitle{Recap}
\begin{itemize}
\item What are axioms, propositions, theorems, claims and proofs?
\item Various theorems we proved in class:


\begin{itemize}
\item If $n$ is a composite integer, then $n$ has a prime divisor less than or equal to $\sqrt{n}$.

\item If $r$ is irrational then $\sqrt{r}$ is irrational.

\item $\sqrt{2}$ is irrational.

\item Well-ordering principle implies induction.

\item $2^n < n!$

\item $4a^3+2b^3=c^3$ does not have roots over $\RingNat$.

\end{itemize}



\item The well ordering principle, induction, and strong induction.
\end{itemize}
You were asked to think about the following two problems:
\begin{itemize}
\item Is $2^n < \frac{n}{2}!$?
\item For every positive integer $n$ there exists another positive integer $k$ such that $n$ is of the form $9k, 9k+1,$ or $9k-1$.
\end{itemize}

\end{frame}

%%%%%%%%%%%%%%%%%%%%%%%%%%%%%%%%%%%%%%%%%%%%%%%%%%%%%%%%%%%%%%%%%%%%%%%%%%%%%%%%
 \begin{frame}[fragile]
\frametitle{Another proof by induction}
\begin{theorem}
For any $n\in \RingNat, n \geq 2$ prove that\\
$\sqrt{2\sqrt{3\sqrt{4\ldots \sqrt{n-1\sqrt{n}}}}} < 3$
\textcolor{blue}{$\Leftarrow \forall 2 \leq i < j, i,j \in \RingNat, f(i,j) < i+1$ \hfill($\ast$)} 
\end{theorem}

A slightly stronger induction hypothesis is required


\begin{proof}
For all $2 \leq i \leq j, i,j \in \RingNat$ let $f(i,j) = \sqrt{i\sqrt{i+1\ldots \sqrt{j}}}$. \\
We will prove a slightly more general statement:\\
For all $2 \leq i \leq j, i,j \in \RingNat, f(i,j) < i+1$\\
\textcolor{blue}{This is more general than the theorem statement we wanted to prove.}

\end{proof}
\end{frame}



%%%%%%%%%%%%%%%%%%%%%%%%%%%%%%%%%%%%%%%%%%%%%%%%%%%%%%%%%%%%%%%%%%%%%%%%%%%%%%%%
 \begin{frame}[fragile]
\frametitle{Another proof by induction}
We prove ($\ast$) by induction on $j-i$.\\

Base case: $j-i=1$. $f(i,i+1) = \sqrt{i\sqrt{i+1}}< i+1$.\\

Induction: 
\begin{align*}
f(i,j+1) &= \sqrt{i\cdot f(i+1,j+1)} 
&\\ 
&< \sqrt{i\cdot (i+2)} \qquad\qquad&\mbox{(by Induction Hypothesis)}\\ 
&\leq i+1 &\mbox{(by AM-GM inequality)} 
\end{align*}
\end{frame}

%%%%%%%%%%%%%%%%%%%%%%%%%%%%%%%%%%%%%%%%%%%%%%%%%%%%%%%%%%%%%%%%%%%%%%%%%%%%%%%%
 \begin{frame}[fragile]
\frametitle{A Bogus Inductive proof}
\begin{theorem}[Bogus, \textcolor{blue}{CW2.2}]
$a\in \RingReal$, $a>0$. Then, $\forall n \in \RingNat$, $a^n = 1$.
\end{theorem}

\begin{proof}[By Strong Induction]
Base case: $n=0$. So $a^n=1$.\\

Induction: $n\rightarrow n+1$. \\

\[
a^{n+1} = \frac{a^n\cdot a^n}{a^{n-1}} = \frac{1\cdot 1}{1} = 1
\]

\begin{center}
{\Large ???}
\end{center}
\end{proof}

\end{frame}

%%%%%%%%%%%%%%%%%%%%%%%%%%%%%%%%%%%%%%%%%%%%%%%%%%%%%%%%%%%%%%%%%%%%%%%%%%%%%%%%
 \begin{frame}[fragile]
\frametitle{Recap}
\begin{itemize}
\item The principle of induction: we proved that $\forall i,j \in \RingNat, ~f(i,j) < i+1,$ where $f(i,j) = \sqrt{i\sqrt{i+1 \ldots \sqrt{j-1 \sqrt{j}}}}$. 

\item Take back message: be careful when proving statements by induction.
\end{itemize}
\end{frame}


%%%%%%%%%%%%%%%%%%%%%%%%%%%%%%%%%%%%%%%%%%%%%%%%%%%%%%%%%%%%%%%%%%%%%%%%%%%%%%%%
 \begin{frame}[fragile]
\frametitle{}
\begin{center}
{\Large \textcolor{brown}{Mathematical Structures}}\\
sets, functions, relations, graphs ...
\end{center}
\end{frame}


%%%%%%%%%%%%%%%%%%%%%%%%%%%%%%%%%%%%%%%%%%%%%%%%%%%%%%%%%%%%%%%%%%%%%%%%%%%%%%%%
 \begin{frame}[fragile]
\frametitle{What are sets?}
A set can be vaguely defined as a collection of objects.\\

But vague definitions can lead to problems.\\


{\bf The barber's dilema}\\

Once upon a time there was a kingdom in which the king ordered the barber to shave only those who 
do not shave themselves! 


Of course, barber could neither shave himself and nor could he not shave himself!\\

This is called a paradox.\\

Why did the paradox arise? 
 -- the king should have excluded the barber from the set of all people.


Cantor was the first person to define sets formally -- finite sets as well as infinite sets, and prove important properties related to sets.\\



\end{frame}

%%%%%%%%%%%%%%%%%%%%%%%%%%%%%%%%%%%%%%%%%%%%%%%%%%%%%%%%%%%%%%%%%%%%%%%%%%%%%%%%
 \begin{frame}[fragile]
\frametitle{What are sets?}

Let $P$ be a property then he said any collection of objects which satisfy property $P$ is a set, i.e.\\
$S = \{x \mid P(x)\}$.

{\bf Russell's paradox:}\\
$A = \{X \mid X \notin X\}$\\

Now if $A \in A$ then $A \notin A$ and if $A \notin A$ then $A \in A$!\\

\textcolor{blue}{(CW)} Can you come up with a set that contains itself?\\




\end{frame}

%%%%%%%%%%%%%%%%%%%%%%%%%%%%%%%%%%%%%%%%%%%%%%%%%%%%%%%%%%%%%%%%%%%%%%%%%%%%%%%%
 \begin{frame}[fragile]
\frametitle{What are sets?}

How to get around this paradox?
\begin{definition}
Start with a few objects {\it defined} as sets. Now if $A$ is a set and $P$ is a property, 
then $S = \{x \in A \mid P(x)\}$ is also a set. 
\end{definition}

Why does this definition get rid of Russell's paradox?

\begin{itemize}
\item Let $P(x) = x \notin x$. Suppose $A$ is a set and let $S = \{x \in A \mid x \notin x\}$.


\begin{itemize}
\item ($S \in S$:) from the definition of $S$, $S \in A$ and $S \notin S$, which is a contradiction.
\item ($S \notin S$:) from the definition, either $S \notin A$ or $S \in S$. But we have assumed that $S \notin S$, therefore it must mean $S \notin A$.  There is no contradiction!
\end{itemize}

\end{itemize}



How to get around Barber's paradox? \textcolor{blue}{(CW)}

\end{frame}

%%%%%%%%%%%%%%%%%%%%%%%%%%%%%%%%%%%%%%%%%%%%%%%%%%%%%%%%%%%%%%%%%%%%%%%%%%%%%%%%
 \begin{frame}[fragile]
\frametitle{Examples and properties}
\begin{itemize}
\item We have already seen sets such as $\RingNat, \RingInt, \RingReal$ etc. 

\item Let $A,B$ be two sets. Their cartesian product, $A\times B$, is defined as \\
$A \times B = \{(a,b) \mid a \in A, b \in B\}$
\item Similarly, union, intersection, symmetric difference are defined as:\\
$A \cup B = \{x \mid a \in A \mbox { or } x \in B\}$\\
$A \cap B = \{x \mid a \in A \mbox { and } x \in B\}$\\
$A \oplus B = \{x \mid (x \in A \wedge x \notin B) \vee (x \in B \wedge x \notin A)\}$\\


\textbf{
$\wedge:$ and\\
$\vee:$ or}



\end{itemize}

\end{frame}


%%%%%%%%%%%%%%%%%%%%%%%%%%%%%%%%%%%%%%%%%%%%%%%%%%%%%%%%%%%%%%%%%%%%%%%%%%%%%%%%
 \begin{frame}[fragile]
\frametitle{Examples and properties}
\begin{itemize}

\item Let $U$ be the universe. The complement of a set $A$ with respect to the universe $U$, denoted as $\bar{A}$ or $A^c = \{x \in U \mid x \notin A\}$.

\item The powerset, $\mathcal{P}(A)$, of a set $A$ is defined to be a collection of all subsets of $A$. \\


Example: Let $A = \{a,b\}$ then $\mathcal{P}(A) = \{ \emptyset, \{a\}, \{b\}, \{a,b\}\}$


\end{itemize}

\end{frame}


%%%%%%%%%%%%%%%%%%%%%%%%%%%%%%%%%%%%%%%%%%%%%%%%%%%%%%%%%%%%%%%%%%%%%%%%%%%%%%%%
 \begin{frame}[fragile]
\frametitle{Infinite sets}
\begin{itemize}
\item We have already seen infinite sets:\\
Examples: $\RingNat, \RingInt, \RingReal, \RingRat$.

\item How do we measure the size of any set? For a set $S$, finite or infinite, $|S|$ denotes the size
of that set. It is also called the {\it cardinality} of the set.

\item For a finite set, $|S|$ equals the number of elements in $S$.

\item What about infinite sets?

\item Given two infinite sets, can we talk about one being {\it bigger} than the other? If so, how?
\end{itemize}
\end{frame}

%%%%%%%%%%%%%%%%%%%%%%%%%%%%%%%%%%%%%%%%%%%%%%%%%%%%%%%%%%%%%%%%%%%%%%%%%%%%%%%%
 \begin{frame}[fragile]
\frametitle{Functions}
\begin{definition}
Let $A,B$ be two sets. A function from $A$ to $B$, $f: A \rightarrow B$, is a set defined as follows:\\
$f = \{(a,b)\mid a \in A, b \in B\}$ with an additional property that if $(a,b) \in f$ and $(a,c) \in f$ then $b=c$. 
\end{definition}

\begin{itemize}
\item Here, $b$ is called an image of $a$, denoted as $f(a)=b$.

\item $Range(f) = \{b \in B \mid  \exists a \in A \mbox{ s.t. } f(a)=b\}$   $\subseteq B$\\


\textbf{$\subseteq:$ Subset of }


\item $Domain(f) =A$


\end{itemize}
\end{frame}

%%%%%%%%%%%%%%%%%%%%%%%%%%%%%%%%%%%%%%%%%%%%%%%%%%%%%%%%%%%%%%%%%%%%%%%%%%%%%%%%
 \begin{frame}[fragile]
\frametitle{Types of functions}
\begin{itemize}
\item Injective function, one-to-one: A function $f: A \rightarrow B$ is said to be injective if $\forall x,y \in A$ if $f(x)=f(y)$ then $x =y$.\\


\begin{itemize}
\item Is $f: \RingInt \rightarrow \RingInt$, defined as $f(n) = n+1$ injective?

\item Is $f: \RingInt \rightarrow \RingInt$, defined as $f(n) = n^2$ injective?

\item What about $f: \RingInt \rightarrow \RingInt$, defined as $f(n) = \sqrt{n}$?

\end{itemize}


\item Surjective function, onto: A function $f: A \rightarrow B$ is said to be surjective if $\forall x\in B$ 
$\exists a \in A$ such that $f(a)=x$. \\


\begin{itemize}
\item Is $f: \RingInt \rightarrow \RingInt$, defined as $f(n) = n+1$ surjective?

\item Is $f: \RingInt \rightarrow \RingInt$, defined as $f(n) = n^2$ surjective?

\item What about $f: \RingReal \rightarrow \RingReal$, defined as $f(x) = 10x - 7$?

\end{itemize}


\item Bijective function: A function is said to be bijective if it is surjective and injective. \\


\begin{itemize}
\item Is $f: \RingInt \rightarrow \RingInt$, defined as $f(n) = n+1$ bijective?

\item Is $f: \RingInt \rightarrow \RingInt$, defined as $f(n) = n^2$ bijective?

\item What about $f: \RingReal \rightarrow \RingReal$, defined as $f(x) = 10x - 7$?

\end{itemize}

\end{itemize}
\end{frame}

%%%%%%%%%%%%%%%%%%%%%%%%%%%%%%%%%%%%%%%%%%%%%%%%%%%%%%%%%%%%%%%%%%%%%%%%%%%%%%%%
 \begin{frame}[fragile]
\frametitle{Back to infinite sets}
We will understand the notion of {\it size of an infinite set} in a relative sense.

\begin{definition}
We say that two sets $A,B$ have the same size if and only if there is a bijection between $A$ and $B$
\end{definition}


Examples

\begin{itemize}
\item Let $E$ be a set of even numbers. There is a bijection between $E$ and $\RingNat$\\


$f(x) = 2x, f: \RingNat \rightarrow E$.


\item There is a bijection $f: \RingInt \rightarrow \RingNat$\\


$f(x) = \left\{\begin{array}{cc}
				-2x & \mbox{ if } x \leq 0\\
				2x-1 & \mbox{otherwise}\\
			\end{array}	
				\right.$


\item There is a bijection $f: \RingNat \times \RingNat \rightarrow \RingNat$\\

%\only<7>{
%$f(x,y) = \left(\sum_{i=1}^{x+y} i \right)+ y+1$
%}
%
%\item There is a bijection $f: \RingNat \times \RingNat \times \RingNat \rightarrow \RingNat$\\
%
%\only<9>{
%\textcolor{blue}{CW}
%}
%
%\item Is there a bijection between $\RingReal$ and $\RingNat?$
%
%\item Is there a bijection between $\RingReal$ and set of all subsets of $\RingNat$?
\end{itemize}
\end{frame}
